\begin{problem}
  Consider the samples in the ``Play-tennis'' dataset from Table~\ref{tab:P02:PlayTennis}. If you calculate the information-gain for all of the attributes of this set, you will observe that the attribute “Outlook” has the largest information-gain, which is equal to 0.246. Therefore, the attribute ”Outlook” is the best heuristic choice for the root node.
\end{problem}

\begin{table}[h]
  \centering
  \caption{Play tennis table}\label{tab:P02:PlayTennis}
  \begin{tabular}{|c||c|c|c|c|c|}
    \hline
    \textbf{Day}  & \textit{Outlook}  & \textit{Temperature} & \textit{Humidity} & \textit{Wind} & \textit{PlayTennis} \\\hline\hline
    D1  & Sunny    & Hot   & High   & Weak   & No \\\hline
    D2  & Sunny    & Hot   & High   & Strong & No \\\hline
    D3  & Overcast & Hot   & High   & Weak   & Yes\\\hline
    D4  & Rain     & Mild  & High   & Weak   & Yes\\\hline
    D5  & Rain     & Cool  & Normal & Weak   & Yes\\\hline
    D6  & Rain     & Cool  & Normal & Strong & No \\\hline
    D7  & Overcast & Cool  & Normal & Strong & Yes\\\hline
    D8  & Sunny    & Mild  & High   & Weak   & No \\\hline
    D9  & Sunny    & Cool  & Normal & Weak   & Yes\\\hline
    D10 & Rain     & Mild  & Normal & Weak   & Yes\\\hline
    D11 & Sunny    & Mild  & Normal & Strong & Yes\\\hline
    D12 & Overcast & Mild  & High   & Strong & Yes\\\hline
    D13 & Overcast & Hot   & Normal & Weak   & Yes\\\hline
    D14 & Rain     & Mild  & High   & Strong & No \\\hline
  \end{tabular}
\end{table}

\begin{subproblem}
  List the labels of the new tree branches below the root node.
\end{subproblem}

``Sunny'', ``Rain'', \& ``Overcast''

\begin{subproblem}
  Which partition of the data will be assigned to each branch by ID3? Please list the sample IDs that will be assigned to each branch.
\end{subproblem}

\begin{table}[h]
  \centering
  \caption{ID3 partition over attribute ``Outlook''}\label{tab:P02:Partition}
  \begin{tabular}{|c|c|}
    \hline
    \textbf{Outlook Value} & \textbf{IDs} \\\hline
    Sunny    & D1, D2, D8, D9, D11 \\\hline
    Overcast & D3, D7, D12, D13 \\\hline
    Rain     & D4, D5, D6, D10, D14 \\\hline
  \end{tabular}
\end{table}

\begin{subproblem}
  Calculate the information gain for the remaining attributes in each branch, and determine which attribute will be chosen as the root of the sub-tree in each branch.
\end{subproblem}

\noindent
{\large \textbf{Case~\#1}: ``Sunny''}

\begin{align*}
  H_{\text{Sunny}} &= -\sum_{x\in PlayTennis} \Pr[x] \log \left(\Pr[x]\right) \\
                   &= - \frac{3}{5} \log\left(\frac{3}{5}\right) - \frac{2}{5} \log\left(\frac{2}{5}\right) \\
                   &\approx 0.971
\end{align*}

\noindent
\textbf{Temperature}: $IG_{\text{Temperature}} \approx 0.971 - 0.4 = \boxed{0.571}$

  \begin{table}[h]
    \centering
    \caption{Temperature information gain}
    \begin{tabular}{c|c|c}
      \hline
      Temperature  & \textit{PlayTennis} & Examples \\\hline\hline
      \multirow{2}{*}{Hot}  & \textit{Yes} & N/A \\
                            & \textit{No}  & D1, D2\\\hline
      \multirow{2}{*}{Mild} & \textit{Yes} & D8 \\
                            & \textit{No}  & D11 \\\hline
      \multirow{2}{*}{Cold} & \textit{Yes} & D9 \\
                            & \textit{No}  & N/A \\\hline
    \end{tabular}
  \end{table}

  \begin{aligncustom}
    H_{\text{Temp}} &= \frac{2}{5} \cdot 0 + \frac{2}{5} \cdot 1 + \frac{1}{5} \cdot 0 \\
                    &= \frac{2}{5}
  \end{aligncustom}

\noindent
\textbf{Humidity}: $IG_{\text{Humidity}} \approx 0.971 - 0 = \boxed{0.971} $

  \begin{table}[h]
    \centering
    \caption{Temperature information gain}
    \begin{tabular}{c|c|c}
      \hline
      Temperature  & \textit{PlayTennis} & Examples \\\hline\hline
      \multirow{2}{*}{High}   & \textit{Yes} & N/A \\
                              & \textit{No}  & D1, D2, D8\\\hline
      \multirow{2}{*}{Normal} & \textit{Yes} & D9, D11 \\
                              & \textit{No}  & N/A \\\hline
    \end{tabular}
  \end{table}

  \begin{aligncustom}
    H_{\text{Humidity}} &= \frac{3}{5} \cdot 0 + \frac{2}{5} \cdot 0 \\
                        &= 0
  \end{aligncustom}

\noindent
\textbf{Wind}: $IG_{\text{Wind}} \approx 0.971 - 0.8 = \boxed{0.171} $

  \begin{table}[h]
    \centering
    \caption{Temperature information gain}
    \begin{tabular}{c|c|c}
      \hline
      Temperature  & \textit{PlayTennis} & Examples \\\hline\hline
      \multirow{2}{*}{Strong} & \textit{Yes} & N/A \\
                              & \textit{No}  & D2 \\\hline
      \multirow{2}{*}{Weak}   & \textit{Yes} & D9, D11 \\
                              & \textit{No}  & D1, D8 \\\hline
    \end{tabular}
  \end{table}

  \begin{aligncustom}
    H_{\text{Humidity}} &= \frac{4}{5} \cdot 1 + \frac{1}{5} \cdot 0 \\
                        &= \frac{4}{5}
  \end{aligncustom}

\begin{center}
  \textbf{\blue{Selected Variable}}: $\boxed{Humidity}$
\end{center}

\noindent
{\large \textbf{Case~\#2}: ``Overcast''}

\begin{align*}
  H_{\text{Overcast}} &= -\sum_{x\in PlayTennis} \Pr[x] \log \left(\Pr[x] \right)\\
                      &= - 0 \log(0) - 1 \log(1) \\
                      &= 0
\end{align*}

Since the entropy of value ``Overcast'' is already~0, there is no remaining information to gain (i.e.,~gain for all remaining variables will be 0).  Therefore, ``Overcast'' is a leaf node for \textit{PlayTennis} ``Yes''.

\noindent
{\large \textbf{Case~\#3}: ``Rain''}

\begin{align*}
  H_{\text{Rain}} &= -\sum_{x\in PlayTennis} \Pr[x] \log \left(\Pr[x]\right) \\
                  &= - \frac{2}{5} \log\left(\frac{2}{5}\right) - \frac{3}{5} \log\left(\frac{3}{5}\right) \\
                  &\approx 0.971
\end{align*}

\noindent
\textbf{Temperature}: $IG_{\text{Temperature}} \approx 0.971 - 0.951 = \boxed{0.020}$

  \begin{table}[h]
    \centering
    \caption{Temperature information gain}
    \begin{tabular}{c|c|c}
      \hline
      Temperature  & \textit{PlayTennis} & Examples \\\hline\hline
      \multirow{2}{*}{Hot}  & \textit{Yes} & N/A \\
                            & \textit{No}  & N/A \\\hline
      \multirow{2}{*}{Mild} & \textit{Yes} & D4, D10 \\
                            & \textit{No}  & D14 \\\hline
      \multirow{2}{*}{Cold} & \textit{Yes} & D5 \\
                            & \textit{No}  & D6 \\\hline
    \end{tabular}
  \end{table}

  \begin{aligncustom}
    H_{\text{Temp}} &= \frac{0}{5} \cdot \infty + \frac{3}{5} \cdot \left( -\frac{1}{3}\log\left(\frac{1}{3}\right)  -\frac{2}{3}\log\left(\frac{2}{3}\right)\right) + \frac{2}{5} \cdot 1 \\
                    &\approx 0 + \frac{3}{5} \cdot 0.918  + \frac{2}{5} \\
                    &\approx 0.951
  \end{aligncustom}

\noindent
\textbf{Humidity}: $IG_{\text{Humidity}} \approx 0.971 - 0.951 = \boxed{0.020} $

  \begin{table}[h]
    \centering
    \caption{Temperature information gain}
    \begin{tabular}{c|c|c}
      \hline
      Temperature  & \textit{PlayTennis} & Examples \\\hline\hline
      \multirow{2}{*}{High}   & \textit{Yes} & D4 \\
                              & \textit{No}  & D14 \\\hline
      \multirow{2}{*}{Normal} & \textit{Yes} & D5, D10 \\
                              & \textit{No}  & D6 \\\hline
    \end{tabular}
  \end{table}

  \begin{aligncustom}
    H_{\text{Humidity}} &= \frac{2}{5} \cdot 1 + \frac{3}{5} \cdot \left( -\frac{1}{3}\log\left(\frac{1}{3}\right)  -\frac{2}{3}\log\left(\frac{2}{3}\right)\right) \\
                    &\approx \frac{2}{5} + \frac{3}{5} \cdot 0.918 \\
                    &\approx 0.951
  \end{aligncustom}

\noindent
\textbf{Wind}: $IG_{\text{Wind}} \approx 0.971 - 0 = \boxed{0.971} $

  \begin{table}[h]
    \centering
    \caption{Temperature information gain}
    \begin{tabular}{c|c|c}
      \hline
      Temperature  & \textit{PlayTennis} & Examples \\\hline\hline
      \multirow{2}{*}{Strong} & \textit{Yes} & N/A \\
                              & \textit{No}  & D6, D14 \\\hline
      \multirow{2}{*}{Weak}   & \textit{Yes} & D4, D5, D10 \\
                              & \textit{No}  & N/A \\\hline
    \end{tabular}
  \end{table}

  \begin{aligncustom}
    H_{\text{Humidity}} &= \frac{2}{5} \cdot 0 + \frac{3}{5} \cdot 0 \\
                        &= 0
  \end{aligncustom}

\begin{center}
  \textbf{\blue{Selected Variable}}: $\boxed{Wind}$
\end{center}
