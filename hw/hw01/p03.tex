\begin{problem}
  Suppose a bank makes loan decisions using two decision trees, one that uses attributes related to credit history and one that uses other demographic attributes. Each decision tree separately classifies a loan applicant as “High Risk” or “Low Risk.” The bank only offers a loan when both decision trees predict “Low Risk.”
\end{problem}

\begin{subproblem}
  Describe an algorithm for converting this pair of decision trees into a single decision tree that makes the same predictions (that is, it predicts non-risky only when both of the original decision trees would have predicted non-risky).
\end{subproblem}

For notational convenience, name the decision tree constructed from credit history as~$T_1$ and the other decision tree as~$T_2$.

A very simple algorithm is to start with~$T_1$ and replace any ``Low Risk'' leaf node in~$T_1$ with~$T_2$.  Clearly, if an example is ``high risk'' in $T_1$, it is still ``High Risk'' in this combined tree.  If the example is ``Low Risk'' in $T_1$, its final label matches the label from $T_2$.  The leaf replacement strategy described previously ensures that paradigm.

One could raise the concern that the same feature may appear more than once in a root-to-leaf path.  While that tree is not minimally compact, it does not affect the correctness.

\begin{subproblem}
  Let $n_1$ and $n_2$ be the number of leaves in the first and second decision trees, respectively. Provide an upper bound on $n$, the number of leaves in the single equivalent decision tree, expressed as a function of $n_1$ and $n_2$.
\end{subproblem}

