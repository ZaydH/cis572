\documentclass{report}

\newcommand{\name}{Zayd Hammoudeh}
\newcommand{\course}{CIS572}
\newcommand{\assnName}{Homework~3: Nearest Neighbors}
\newcommand{\dueDate}{May~2, 2019}

\usepackage[margin=1in]{geometry}
\usepackage[skip=4pt]{caption}      % ``skip'' sets the spacing between the figure and the caption.
\usepackage{tikz}
\usetikzlibrary{arrows.meta,decorations.markings,shadows,positioning,calc}
\usepackage{pgfplots}               % Needed for plotting
\pgfplotsset{compat=newest}
\usepgfplotslibrary{fillbetween}    % Allow for highlighting under a curve
\usepackage{amsmath}                % Allows for piecewise functions using the ``cases'' construct
\usepackage{siunitx}                % Allows for ``S'' alignment in table to align by decimal point

\usepackage{multirow}
\usepackage{multicol}

\usepackage[obeyspaces,spaces]{url} % Used for typesetting with the ``path'' command
\usepackage[hidelinks]{hyperref}    % Make the cross references clickable hyperlinks
\usepackage[bottom]{footmisc}       % Prevents the table going below the footnote
\usepackage{nccmath}                % Needed in the workaround for the ``aligncustom'' environment
\usepackage{amssymb}                % Used for black QED symbol
\usepackage{bm}                     % Allows for bolding math symbols.
\usepackage{tabto}                  % Allows to tab to certain point on a line
\usepackage{float}
\usepackage{subcaption}             % Allows use of the ``subfigure'' environment
\usepackage{enumerate}              % Allow enumeration other than just numbers
\usepackage{pdfpages}
\usepackage{placeins}

\usepackage[noend]{algpseudocode}
\usepackage[Algorithm,ruled]{algorithm}
\algnewcommand\algorithmicforeach{\textbf{for each}}
\algdef{S}[FOR]{ForEach}[1]{\algorithmicforeach\ #1\ \algorithmicdo}

\usepackage{mathtools} % for "\DeclarePairedDelimiter" macro
\DeclarePairedDelimiter{\floor}{\lfloor}{\rfloor}
\DeclarePairedDelimiter{\ceil}{\lceil}{\rceil}
\DeclarePairedDelimiter{\abs}{\lvert}{\rvert}

%---------------------------------------------------%
%     Define Distances Used for Document Margins    %
%---------------------------------------------------%

\newcommand{\hangindentdistance}{1cm}
\newcommand{\defaultleftmargin}{0.25in}
\newcommand{\questionleftmargin}{-.5in}

\setlength{\parskip}{1em}
\setlength{\oddsidemargin}{\defaultleftmargin}

%---------------------------------------------------%
%      Configure the Document Header and Footer     %
%---------------------------------------------------%

% Set up page formatting
\usepackage{todonotes}
\usepackage{fancyhdr}                   % Used for every page footer and title.
\pagestyle{fancy}
\fancyhf{}                              % Clears both the header and footer
\renewcommand{\headrulewidth}{0pt}      % Eliminates line at the top of the page.
\fancyfoot[LO]{\course\ --- \assnName}   % Left
\fancyfoot[CO]{\thepage}                % Center
\fancyfoot[RO]{\name}                   % Right

%---------------------------------------------------%
%           Define the Title Page Entries           %
%---------------------------------------------------%

\title{\textbf{\course\ --- \assnName}}
\author{\name}

%---------------------------------------------------%
% Define the Environments for the Problem Inclusion %
%---------------------------------------------------%

\usepackage{scrextend}
\newcounter{problemCount}
\setcounter{problemCount}{0}  % Reset the subproblem counter

\newcounter{subProbCount}[problemCount]   % Reset subProbCount any time problemCount changes.
\renewcommand{\thesubProbCount}{\alph{subProbCount}}  % Make it so the counter is referenced as a number

\newenvironment{problemshell}{
  \begin{addmargin}[\questionleftmargin]{0em}
    \par%
    \medskip%
    \leftskip=0pt\rightskip=0pt%
    \setlength{\parindent}{0pt}
    \bfseries%
  }
  {%
    \par\medskip%
  \end{addmargin}
}
\newenvironment{problem}
{%
  \refstepcounter{problemCount} % Increment the subproblem counter.  This must be before the exercise to ensure proper numbering of claims and lemmas.
  \begin{problemshell}
    \noindent \textit{Exercise~\#\arabic{problemCount}} \\
  }
  {
  \end{problemshell}
  %  \setcounter{subProbCount}{0} % Reset the subproblem counter
}
\newenvironment{subproblem}
{%
  \begin{problemshell}
    \refstepcounter{subProbCount} % Increment the subproblem counter
    \setlength{\leftskip}{\hangindentdistance}
    % Print the subproblem count and offset to the left
    \hspace{-\hangindentdistance}(\alph{subProbCount}) \tabto{0pt}
  }
  {
  \end{problemshell}
}

% Change interline spacing.
\renewcommand{\baselinestretch}{1.1}
\newenvironment{aligncustom}
{ \csname align*\endcsname % Need to do this instead of \begin{align*} because of LaTeX bug.
  \centering
}
{
  \csname endalign*\endcsname
}

%---------------------------------------------------%
%      Define commands for coloring the text.       %
%---------------------------------------------------%

\newcommand{\red}[1]{{\color{red} #1}}
\newcommand{\blue}[1]{{\color{blue} #1}}
\newcommand{\green}[1]{{\color{green} #1}}

%---------------------------------------------------%
% Define the Environments for the Problem Inclusion %
%---------------------------------------------------%

\usepackage{amsthm}       % Allows use of the ``proof'' environment.

% Number lemmas and claims using the problem count
\newtheorem{claim}{Claim}[problemCount]
\newtheorem{lemma}{Lemma}[problemCount]

%---------------------------------------------------%
%       Define commands related to managing         %
%    floats (e.g., images) across multiple pages    %
%---------------------------------------------------%

\usepackage{placeins}     % Allows \FloatBarrier

% Prevent preceding floats going to this page
\newcommand{\floatnewpage}{\FloatBarrier\newpage}

% Add the specified input file and prevent any floated figures/tables going onto the same page as new input
\newcommand{\problemFile}[1]{
  \floatnewpage%
  \input{#1}
}



\begin{document}
  \maketitle

  \noindent
  \textbf{Name}: \name\\
  \textbf{Course}: \course\\
  \textbf{Assignment}: \assnName\\
  \textbf{Due Date}: \dueDate%

  \noindent
  % \textbf{Other Student Discussions}: I discussed the problems in this homework with the following student(s) below.  All write-ups were prepared independently.
  % \vspace{-1em}
  % \begin{itemize}
  %   \item Steven Walton
  % \end{itemize}

  \newpage
  \newcommand{\featureOp}{normalized}

\begin{problem}
An analyst wants to classify a number of customers based on some given attributes: total number of accounts, credit utilization (amount of used credit divided by total credit amount), percentage of on-time payments, age of credit history, and inquiries (number of time that the customer requested a new credit account, whether accepted or not). The analyst acquired some labeled information as shown in the following table:

\begin{table}[h]
  \centering
  \caption{Raw training data for problem~\#1}\label{tab:P01:RawTrain}
  \begin{tabular}{|c||c|c|c|c|c|c|}
    \hline
    ID & Total    & Utilization & Payment & Age of History & Inquiries & Label \\
       & Accounts &             & History & (days)         &           & \\\hline\hline
    1 & 8 & 15\% & 100\% & 1000 & 5 & GOOD \\\hline
    2 & 15 & 19\% & 90\% & 2500 & 8 & BAD \\\hline
    3 & 10 & 35\% & 100\% & 500 & 10 & BAD \\\hline
    4 & 11 & 40\% & 95\% & 2000 & 6 & BAD \\\hline
    5 & 12 & 10\% & 99\% & 3000 & 6 & GOOD \\\hline
    6 & 18 & 15\% & 100\% & 2000 & 5 & GOOD \\\hline
    7 & 3 & 21\% & 100\% & 1500 & 7 & BAD \\\hline
    8 & 14 & 4\% & 100\% & 3500 & 5 & GOOD \\\hline
    9 & 13 & 5\% & 100\% & 3000 & 3 & GOOD \\\hline
    10 & 6 & 25\% & 94\% & 2800 & 9 & BAD \\\hline
  \end{tabular}
\end{table}


Consider the following three accounts to be labeled:

\begin{table}[h]
  \centering
  \caption{Raw unlabeled accounts data for problem~\#1}\label{tab:P01:RawUnlabel}
  \begin{tabular}{|c|c|c|c|c|c|}
    \hline
    Total    & Utilization & Payment & Age of History & Inquiries & Label \\
    Accounts &             & History & (days)         &           & \\\hline\hline
    20 & 50\% & 90\% & 4500 & 12 & P1 \\\hline
    8 & 10\% & 100\% & 550 & 4 & P2 \\\hline
    9 & 13\% & 99\% & 3000 & 6 & P3 \\\hline
  \end{tabular}
\end{table}
\end{problem}

\begin{subproblem}
  Before using nearest neighbor methods to make predictions, how would you recommend processing or transforming the data? Why? Make any changes you think appropriate to the data before continuing on to the next two parts.
\end{subproblem}

K-NN relies on a distance metric to quantify similarity between two examples.  If features are not changed to a consistent scales, necessarily some features will have inherently higher weight than other --- in particular those with significant higher magnitude.  In some cases, that may be advantageous but doing so blindly is generally dangerous and a bad idea since it may yield dubious classifications.

The two most common approaches for transforming features are \textit{standardization} and \textit{normalization}.  Normalization is sensitive to outliers but creates relatively well-bounded feature ranges.  Since this data appears free of outliers, it is used for this problem. The \featureOp\ data for Tables~\ref{tab:P01:RawTrain} and~\ref{tab:P01:RawUnlabel} are in Tables~\ref{tab:P01:TransformTrain} and~\ref{tab:P01:TransformUnlabel} respectively.

Observe that some values in Table~\ref{tab:P01:TransformUnlabel} are outside the range~$[0,1]$.  This is because the minimum and maximum values used for normalization derive exclusively from the training set, and the validation set values may be outside the training set's range.

\begin{table}[ht]
  \centering
  \caption{Problem~\#\arabic{problemCount} \featureOp\ training data}\label{tab:P01:TransformTrain}
  \begin{tabular}{|c||c|c|c|c|c|c|}
    \hline
    ID & Total    & Utilization & Payment & Age of History & Inquiries & Label \\
       & Accounts &             & History & (days)         &           & \\\hline\hline
    1 & 0.333 & 0.306 & 1.000 & 0.167 & 0.286 & GOOD \\\hline
2 & 0.800 & 0.417 & 0.000 & 0.667 & 0.714 & BAD \\\hline
3 & 0.467 & 0.861 & 1.000 & 0.000 & 1.000 & BAD \\\hline
4 & 0.533 & 1.000 & 0.500 & 0.500 & 0.429 & BAD \\\hline
5 & 0.600 & 0.167 & 0.900 & 0.833 & 0.429 & GOOD \\\hline
6 & 1.000 & 0.306 & 1.000 & 0.500 & 0.286 & GOOD \\\hline
7 & 0.000 & 0.472 & 1.000 & 0.333 & 0.571 & BAD \\\hline
8 & 0.733 & 0.000 & 1.000 & 1.000 & 0.286 & GOOD \\\hline
9 & 0.667 & 0.028 & 1.000 & 0.833 & 0.000 & GOOD \\\hline
10 & 0.200 & 0.583 & 0.400 & 0.767 & 0.857 & BAD \\\hline
  \end{tabular}
\end{table}

\begin{table}[ht]
  \centering
  \caption{Problem~\#\arabic{problemCount} \featureOp\ unlabeled data}\label{tab:P01:TransformUnlabel}
  \begin{tabular}{|c|c|c|c|c||c|}
    \hline
    Total    & Utilization & Payment & Age of History & Inquiries & ID \\
    Accounts &             & History & (days)         &           & \\\hline\hline
    1.133 & 1.278 & 0.000 & 1.333 & 1.286 & P1 \\\hline
0.333 & 0.167 & 1.000 & 0.017 & 0.143 & P2 \\\hline
0.400 & 0.250 & 0.900 & 0.833 & 0.429 & P3 \\\hline
  \end{tabular}
\end{table}

\FloatBarrier%
\begin{subproblem}
  What are the predicted labels P1, P2, and P3 using 1-NN with $L_1$ distance? Assume that percentages are represented as their corresponding decimal numbers, so 95\% = 0.95. Show your work.
\end{subproblem}

The predicted labels for the unlabeled data in Table~\ref{tab:P01:TransformUnlabel} is shown in Table~\ref{tab:P01:b:PredictedLabels}.  The calculations of the $L_1$ loss for P1, P2, and P3 are in Tables~\ref{tab:P01:b:P1},~\ref{tab:P01:b:P2}, and~\ref{tab:P01:b:P3} respectively.  Observe that for each unlabeled example, the minimum $L_1$ loss is in bold.

\begin{table}[h]
  \centering
  \caption{Predicted labels for Problem~\arabic{problemCount}(\alph{subProbCount})}\label{tab:P01:b:PredictedLabels}
  \begin{tabular}{|c|c|}
    \hline
    Example & Predicted Label \\\hline\hline
    1 & BAD \\\hline2 & GOOD \\\hline3 & GOOD \\\hline
  \end{tabular}
\end{table}

\@for\id:={1,2,3}\do{%
  \begin{table}[ht]
    \centering
    \caption{$L_1$ loss by feature for the \featureOp\ P\id\ data}\label{tab:P01:b:P\id}
    \begin{tabular}{|c||c|c|c|c|c||c|}
      \hline
      ID & Total    & Utilization & Payment & Age of History & Inquiries & Total \\
         & Accounts &             & History & (days)         &           & Loss \\\hline\hline
      \input{src/tables/p01_part_b_p\id_err}
    \end{tabular}
  \end{table}
}

\FloatBarrier%
\begin{subproblem}
  Keep the information of customers 7, 8, 9, and 10 as validation data, and find the best K value for the K-NN algorithm. If the best value of $K$ is not equal to 1, find the new predictions for P1, P2, and P3. Show your work.
\end{subproblem}

For even valued~$K$, label ties are possible.  In this homework, all ties were broken in favor of ``BAD.''  All tables and calculations are generated with a Python3 script that can be provided on request.

Table~\ref{tab:P01:c:AccVsK} shows the accuracy on the validation set for all possible values of $K$ with the best value~(2) shown in \textbf{bold}.  The $L_1$ loss for each feature on each of the four validation examples is shown in Tables~\ref{tab:P01:c:Valid7}--\ref{tab:P01:c:Valid10}.  Tables~\ref{tab:P01:c:BestNeighK1} through~\ref{tab:P01:c:BestNeighK6} show the selected ``nearest neighbors'' and predicted label for each validation example for $K$ equal to~1 through~6 respectively.

The predicted labels along with the nearest neighbors for~P1,~P2, and~P3 using the best $K$~(2) for the validation set is shown in Table~\ref{tab:P01:c:UnlabelBestK}.  Note that the entire training set in Table~\ref{tab:P01:TransformTrain} is used when labeling P1, P2, and P3, not just the six samples used to determine the best~$K$.

\begin{table}[h]
  \centering
  \caption{Validation accuracy for different values of $K$}\label{tab:P01:c:AccVsK}
  \begin{tabular}{|c|c|}
    \hline
    $K$ & Accuracy \\\hline\hline
    1 & 0.750 \\\hline
2 & \textbf{1.000} \\\hline
3 & 0.750 \\\hline
4 & 0.750 \\\hline
5 & 0.750 \\\hline
6 & 0.500 \\\hline
  \end{tabular}
\end{table}

\begin{table}[h]
  \centering
  \caption{Predicted labels for the unlabeled set with the best $K$}\label{tab:P01:c:UnlabelBestK}
  \begin{tabular}{|c||c|c|}
    \hline
    ID & Best        & Predicted \\
       & Neighbor(s) & Label  \\\hline\hline
    P1 & 2, 10& BAD \\\hline
P2 & 1, 7& BAD \\\hline
P3 & 5, 8& GOOD \\\hline
  \end{tabular}
\end{table}

\@for\id:={7,8,9,10}\do{%
  \begin{table}[ht]
    \centering
    \caption{$L_1$ loss by feature for the \featureOp\ ID \#\id\ validation data}\label{tab:P01:c:Valid\id}
    \begin{tabular}{|c||c|c|c|c|c||c|}
      \hline
      ID & Total    & Utilization & Payment & Age of History & Inquiries & Total \\
         & Accounts &             & History & (days)         &           & Loss \\\hline\hline
      \input{src/tables/p01_c_id_\id_loss}
    \end{tabular}
  \end{table}
}

\@for\id:={1,2,3,4,5,6}\do{%
  \begin{table}[ht]
    \centering
    \caption{Nearest neighbors, predicted label, and actual label for validation examples with $K=\id$}\label{tab:P01:c:BestNeighK\id}
    \begin{tabular}{|c||c|c|c|}
      \hline
      ID & Best      & Predicted & Actual \\
         & Neighbors & Label     & Label \\\hline\hline
      \input{src/tables/p01_c_best_neighbors_k=\id}
    \end{tabular}
  \end{table}
}

  \begin{problem}
  Consider the samples in the ``Play-tennis'' dataset from Table~\ref{tab:P02:PlayTennis}. If you calculate the information-gain for all of the attributes of this set, you will observe that the attribute “Outlook” has the largest information-gain, which is equal to 0.246. Therefore, the attribute ”Outlook” is the best heuristic choice for the root node.
\end{problem}

\begin{table}[h]
  \centering
  \caption{Play tennis table}\label{tab:P02:PlayTennis}
  \begin{tabular}{|c||c|c|c|c|c|}
    \hline
    \textbf{Day}  & \textit{Outlook}  & \textit{Temperature} & \textit{Humidity} & \textit{Wind} & \textit{PlayTennis} \\\hline\hline
    D1  & Sunny    & Hot   & High   & Weak   & No \\\hline
    D2  & Sunny    & Hot   & High   & Strong & No \\\hline
    D3  & Overcast & Hot   & High   & Weak   & Yes\\\hline
    D4  & Rain     & Mild  & High   & Weak   & Yes\\\hline
    D5  & Rain     & Cool  & Normal & Weak   & Yes\\\hline
    D6  & Rain     & Cool  & Normal & Strong & No \\\hline
    D7  & Overcast & Cool  & Normal & Strong & Yes\\\hline
    D8  & Sunny    & Mild  & High   & Weak   & No \\\hline
    D9  & Sunny    & Cool  & Normal & Weak   & Yes\\\hline
    D10 & Rain     & Mild  & Normal & Weak   & Yes\\\hline
    D11 & Sunny    & Mild  & Normal & Strong & Yes\\\hline
    D12 & Overcast & Mild  & High   & Strong & Yes\\\hline
    D13 & Overcast & Hot   & Normal & Weak   & Yes\\\hline
    D14 & Rain     & Mild  & High   & Strong & No \\\hline
  \end{tabular}
\end{table}

\begin{subproblem}
  List the labels of the new tree branches below the root node.
\end{subproblem}

``Sunny'', ``Rain'', \& ``Overcast''

\begin{subproblem}
  Which partition of the data will be assigned to each branch by ID3? Please list the sample IDs that will be assigned to each branch.
\end{subproblem}

\begin{table}[h]
  \centering
  \caption{ID3 partition over attribute ``Outlook''}\label{tab:P02:Partition}
  \begin{tabular}{|c|c|}
    \hline
    \textbf{Outlook Value} & \textbf{IDs} \\\hline
    Sunny    & D1, D2, D8, D9, D11 \\\hline
    Overcast & D3, D7, D12, D13 \\\hline
    Rain     & D4, D5, D6, D10, D14 \\\hline
  \end{tabular}
\end{table}

\begin{subproblem}
  Calculate the information gain for the remaining attributes in each branch, and determine which attribute will be chosen as the root of the sub-tree in each branch.
\end{subproblem}

\noindent
{\large \textbf{Case~\#1}: ``Sunny''}

\begin{align*}
  H_{\text{Sunny}} &= -\sum_{x\in PlayTennis} \Pr[x] \log \left(\Pr[x]\right) \\
                   &= - \frac{3}{5} \log\left(\frac{3}{5}\right) - \frac{2}{5} \log\left(\frac{2}{5}\right) \\
                   &\approx 0.971
\end{align*}

\noindent
\textbf{Temperature}: $IG_{\text{Temperature}} \approx 0.971 - 0.4 = \boxed{0.571}$

  \begin{table}[h]
    \centering
    \caption{Temperature information gain}
    \begin{tabular}{c|c|c}
      \hline
      Temperature  & \textit{PlayTennis} & Examples \\\hline\hline
      \multirow{2}{*}{Hot}  & \textit{Yes} & N/A \\
                            & \textit{No}  & D1, D2\\\hline
      \multirow{2}{*}{Mild} & \textit{Yes} & D8 \\
                            & \textit{No}  & D11 \\\hline
      \multirow{2}{*}{Cold} & \textit{Yes} & D9 \\
                            & \textit{No}  & N/A \\\hline
    \end{tabular}
  \end{table}

  \begin{aligncustom}
    H_{\text{Temp}} &= \frac{2}{5} \cdot 0 + \frac{2}{5} \cdot 1 + \frac{1}{5} \cdot 0 \\
                    &= \frac{2}{5}
  \end{aligncustom}

\noindent
\textbf{Humidity}: $IG_{\text{Humidity}} \approx 0.971 - 0 = \boxed{0.971} $

  \begin{table}[h]
    \centering
    \caption{Temperature information gain}
    \begin{tabular}{c|c|c}
      \hline
      Temperature  & \textit{PlayTennis} & Examples \\\hline\hline
      \multirow{2}{*}{High}   & \textit{Yes} & N/A \\
                              & \textit{No}  & D1, D2, D8\\\hline
      \multirow{2}{*}{Normal} & \textit{Yes} & D9, D11 \\
                              & \textit{No}  & N/A \\\hline
    \end{tabular}
  \end{table}

  \begin{aligncustom}
    H_{\text{Humidity}} &= \frac{3}{5} \cdot 0 + \frac{2}{5} \cdot 0 \\
                        &= 0
  \end{aligncustom}

\noindent
\textbf{Wind}: $IG_{\text{Wind}} \approx 0.971 - 0.8 = \boxed{0.171} $

  \begin{table}[h]
    \centering
    \caption{Temperature information gain}
    \begin{tabular}{c|c|c}
      \hline
      Temperature  & \textit{PlayTennis} & Examples \\\hline\hline
      \multirow{2}{*}{Strong} & \textit{Yes} & N/A \\
                              & \textit{No}  & D2 \\\hline
      \multirow{2}{*}{Weak}   & \textit{Yes} & D9, D11 \\
                              & \textit{No}  & D1, D8 \\\hline
    \end{tabular}
  \end{table}

  \begin{aligncustom}
    H_{\text{Humidity}} &= \frac{4}{5} \cdot 1 + \frac{1}{5} \cdot 0 \\
                        &= \frac{4}{5}
  \end{aligncustom}

\begin{center}
  \textbf{\blue{Selected Variable}}: $\boxed{Humidity}$
\end{center}

\noindent
{\large \textbf{Case~\#2}: ``Overcast''}

\begin{align*}
  H_{\text{Overcast}} &= -\sum_{x\in PlayTennis} \Pr[x] \log \left(\Pr[x] \right)\\
                      &= - 0 \log(0) - 1 \log(1) \\
                      &= 0
\end{align*}

Since the entropy of value ``Overcast'' is already~0, there is no remaining information to gain (i.e.,~gain for all remaining variables will be 0).  Therefore, ``Overcast'' is a leaf node for \textit{PlayTennis} ``Yes''.

\noindent
{\large \textbf{Case~\#3}: ``Rain''}

\begin{align*}
  H_{\text{Rain}} &= -\sum_{x\in PlayTennis} \Pr[x] \log \left(\Pr[x]\right) \\
                  &= - \frac{2}{5} \log\left(\frac{2}{5}\right) - \frac{3}{5} \log\left(\frac{3}{5}\right) \\
                  &\approx 0.971
\end{align*}

\noindent
\textbf{Temperature}: $IG_{\text{Temperature}} \approx 0.971 - 0.951 = \boxed{0.020}$

  \begin{table}[h]
    \centering
    \caption{Temperature information gain}
    \begin{tabular}{c|c|c}
      \hline
      Temperature  & \textit{PlayTennis} & Examples \\\hline\hline
      \multirow{2}{*}{Hot}  & \textit{Yes} & N/A \\
                            & \textit{No}  & N/A \\\hline
      \multirow{2}{*}{Mild} & \textit{Yes} & D4, D10 \\
                            & \textit{No}  & D14 \\\hline
      \multirow{2}{*}{Cold} & \textit{Yes} & D5 \\
                            & \textit{No}  & D6 \\\hline
    \end{tabular}
  \end{table}

  \begin{aligncustom}
    H_{\text{Temp}} &= \frac{0}{5} \cdot \infty + \frac{3}{5} \cdot \left( -\frac{1}{3}\log\left(\frac{1}{3}\right)  -\frac{2}{3}\log\left(\frac{2}{3}\right)\right) + \frac{2}{5} \cdot 1 \\
                    &\approx 0 + \frac{3}{5} \cdot 0.918  + \frac{2}{5} \\
                    &\approx 0.951
  \end{aligncustom}

\noindent
\textbf{Humidity}: $IG_{\text{Humidity}} \approx 0.971 - 0.951 = \boxed{0.020} $

  \begin{table}[h]
    \centering
    \caption{Temperature information gain}
    \begin{tabular}{c|c|c}
      \hline
      Temperature  & \textit{PlayTennis} & Examples \\\hline\hline
      \multirow{2}{*}{High}   & \textit{Yes} & D4 \\
                              & \textit{No}  & D14 \\\hline
      \multirow{2}{*}{Normal} & \textit{Yes} & D5, D10 \\
                              & \textit{No}  & D6 \\\hline
    \end{tabular}
  \end{table}

  \begin{aligncustom}
    H_{\text{Humidity}} &= \frac{2}{5} \cdot 1 + \frac{3}{5} \cdot \left( -\frac{1}{3}\log\left(\frac{1}{3}\right)  -\frac{2}{3}\log\left(\frac{2}{3}\right)\right) \\
                    &\approx \frac{2}{5} + \frac{3}{5} \cdot 0.918 \\
                    &\approx 0.951
  \end{aligncustom}

\noindent
\textbf{Wind}: $IG_{\text{Wind}} \approx 0.971 - 0 = \boxed{0.971} $

  \begin{table}[h]
    \centering
    \caption{Temperature information gain}
    \begin{tabular}{c|c|c}
      \hline
      Temperature  & \textit{PlayTennis} & Examples \\\hline\hline
      \multirow{2}{*}{Strong} & \textit{Yes} & N/A \\
                              & \textit{No}  & D6, D14 \\\hline
      \multirow{2}{*}{Weak}   & \textit{Yes} & D4, D5, D10 \\
                              & \textit{No}  & N/A \\\hline
    \end{tabular}
  \end{table}

  \begin{aligncustom}
    H_{\text{Humidity}} &= \frac{2}{5} \cdot 0 + \frac{3}{5} \cdot 0 \\
                        &= 0
  \end{aligncustom}

\begin{center}
  \textbf{\blue{Selected Variable}}: $\boxed{Wind}$
\end{center}

  \begin{problem}
  Suppose a bank makes loan decisions using two decision trees, one that uses attributes related to credit history and one that uses other demographic attributes. Each decision tree separately classifies a loan applicant as “High Risk” or “Low Risk.” The bank only offers a loan when both decision trees predict “Low Risk.”
\end{problem}

\begin{subproblem}
  Describe an algorithm for converting this pair of decision trees into a single decision tree that makes the same predictions (that is, it predicts non-risky only when both of the original decision trees would have predicted non-risky).
\end{subproblem}

For notational convenience, name the decision tree constructed from credit history as~$T_1$ and the other decision tree as~$T_2$.

A very simple algorithm is to start with~$T_1$ and replace any ``Low Risk'' leaf node in~$T_1$ with~$T_2$.  Clearly, if an example is ``high risk'' in $T_1$, it is still ``High Risk'' in this combined tree.  If the example is ``Low Risk'' in $T_1$, its final label matches the label from $T_2$.  The leaf replacement strategy described previously ensures that paradigm.

One could raise the concern that the same feature may appear more than once in a root-to-leaf path.  While that tree is not minimally compact, it does not affect the correctness.

\begin{subproblem}
  Let $n_1$ and $n_2$ be the number of leaves in the first and second decision trees, respectively. Provide an upper bound on $n$, the number of leaves in the single equivalent decision tree, expressed as a function of $n_1$ and $n_2$.
\end{subproblem}


  \begin{problem}
  The multiclass perceptron maintains a weight vector and bias for each class: $(w_1, b_1),(w_2, b_2),\ldots,(w_k, b_k)$. When it makes an incorrect prediction, it adjusts the weight vectors of both the predicted class $\hat{y}$ and the correct class $y$:

  \begin{aligncustom}
    w_y &\leftarrow w_y + x \\
    b_y &\leftarrow b_y + 1 \\
    w_{\hat{y}} &\leftarrow w_{\hat{y}} - x \\
    b_{\hat{y}} &\leftarrow b_{\hat{y}} - 1
  \end{aligncustom}

  Prove that the standard perceptron is equivalent to the multiclass perceptron when there are only 2~classes. In other words, show that the two methods always make the same predictions when given the same sequence of training examples.
\end{problem}

\newcommand{\cA}{_{+}}
\newcommand{\cB}{_{-}}
\newcommand{\subC}{_{*}}
\newcommand{\tS}[1]{^{(#1)}}

  The decision function for a two-class perceptron is shown in Eq.~\eqref{eq:P04:BaseDecision}.

  \begin{equation}\label{eq:P04:BaseDecision}
    \hat{y}(x) = \begin{cases}
                   y\cA & w\cA \cdot x + b\cA > w\cB \cdot x + b\cB \\
                   y\cB & \text{Otherwise}
                 \end{cases}
  \end{equation}

  \noindent
  It can be simplified to the formulation in Eq.~\eqref{eq:P04:SimpleDecision}.  Note that $w\subC = w\cA - w\cB$ and $b\subC = b\cA - b\cB$.  The following proof exclusively uses this simplified formulation.

  \begin{equation}\label{eq:P04:SimpleDecision}
    \hat{y}(x) = \begin{cases}
                    y\cA & w\subC \cdot x + b\subC > 0 \\
                    y\cB & \text{Otherwise}
                 \end{cases}
  \end{equation}

\begin{proof}
  \textit{By induction}.  The proof demonstrates that for every training example, $x\tS{t}$ where $t=1,2,\ldots$, the decision function of the two-class and standard perceptrons are identical making the two learners necessarily identical.

  \noindent
\textbf{Base Case}:  There are two primary bases cases.  The case of $t=0$ is added for completeness.

\begin{enumerate}[(a)]
  \item $t=0$ --- Before Any Training Examples.  $w\tS{0}\cA=w\tS{0}\cB=w\tS{0}\cA-w\tS{0}\cB=w\tS{0}\subC=\vec{0}$, where $\vec{0}$ is the zero vector.  $b\tS{0}\cA$, $b\tS{0}\cB$, and $b\tS{0}\subC$ are similarly~$0$.  This is identical to the standard perceptron.

  \item $t=1$ --- \textit{Correct} classification.  $w\tS{t+1}\subC=w\tS{t}\subC + x\tS{t} - x\tS{t} = w\tS{t}$.  Therefore, the weight vector is unchanged.  Trivially, the offset $b\subC$ is also unchanged.  This matches the standard perceptron.  Observe that for \textit{all} $t\geq1$, a correct classification never changes $w\subC$ and $b\subC$ so for brevity correct classification is not discussed in the inductive step.

  \item $t=1$ --- \textit{Incorrect} classification.  $w\tS{1}\subC = x\tS{1} - \left(-x\tS{1}\right) = 2x\tS{1}$ for a misclassified positive example, and $w\tS{1}= -x\tS{1} - x\tS{1} = -2x\tS{1}$ for a misclassified negative example.  $b\subC$ is~$2$ and~$-2$ for a misclassified positive and negative respectively.

    Eq.~\eqref{eq:P04:T1Pos} shows the decision function $x\tS{1}$ being a misclassified positive example. The distributed~$2$ has no effect on the classification and can be ignored/divided out.  Therefore, the decision function is again the same as the standard perceptron.  The formulation for a misclassified negative is not shown since it is essentially the same as the positive case with the only difference being $x\tS{1}$ and $1$ in Eq.~\eqref{eq:P04:T1Pos} are negated.
\end{enumerate}

\begin{equation}\label{eq:P04:T1Pos}
  \hat{y}\tS{1}(x) = \begin{cases}
                       y\cA & 2\left(x\tS{1}\cdot x + 1\right) > 0 \\
                       y\cB & \text{Otherwise}
                     \end{cases}
\end{equation}

\noindent
\textbf{Assume}: For $t=k$, standard \& two-class perceptrons have same the decision function (see Eq.~\eqref{eq:P04:Tk}).

\begin{equation}\label{eq:P04:Tk}
  \hat{y}\tS{k}(x) = \begin{cases}
                       y\cA & 2\left(w\tS{k}\subC \cdot x + b\tS{k}\subC\right) > 0 \\
                       y\cB & \text{Otherwise}
                     \end{cases}
\end{equation}

\noindent
Eq.~\eqref{eq:P04:Tk} uses the ``distributed-2'' notation similar to Eq.~\eqref{eq:P04:T1Pos} for the two-class perceptron.

\noindent
\textbf{Inductive Step}: $t=k+1$. If example $x\tS{k+1}$ is positive and misclassified, the decision function is updated as shown in Eq.~\eqref{eq:P04:TkPlus1}. This is again identical to the standard perceptron.

\begin{equation}\label{eq:P04:TkPlus1}
  \hat{y}\tS{k+1}(x) = \begin{cases}
    y\cA & 2\bigg(\left(w\tS{k}\subC + x\tS{k+1}\right)\cdot x + \left(b\tS{k}\subC+1\right)\bigg) > 0 \\
                         y\cB & \text{Otherwise}
                       \end{cases}
\end{equation}

\noindent
Proving the subcase of $k+1$ being a misclassified negative example follows this same procedure and is not shown for brevity.
\end{proof}

  \begin{problem}
Given a set of data points $x_1,\ldots,x_n$, we can define the convex hull to be the set of all points $x$ given by

\[x = \sum_{i=1}^{n} \alpha_i x_i \]

\noindent
where $\alpha_{i} \geq 0$ and $\sum_{i=1}^{n} \alpha_i = 1$. Consider a second set of points $z_1,\ldots,z_m$ together with their corresponding convex hull. By definition, the two sets of points will be linearly separable if there exists a vector $w$ and a scalar $b$ such that $w^{T}x_i + b > 0$ for all $x_i$ and $w^{T}z_i + b < 0$ for all $z_i$.

Prove that the two sets of points are linearly separable if their convex hulls do not intersect.
\end{problem}

\begin{lemma}
  $\mathbf{z}$ \& $\mathbf{x}$ do not intersect $\Rightarrow$ $\mathbf{z}$ and $\mathbf{x}$ linearly separable.
\end{lemma}

\begin{proof}

  Define $\mathbf{x}$ and $\mathbf{z}$ as the convex hulls enumerated above. Define $x_c$ and $z_c$ such that:

  \[ \argmin_{x_c \in \mathbf{x}, z_c \in \mathbf{z}} \norm{x_c - z_c} \text{.} \]

  \noindent
  This distance is strictly positive since $\mathbf{x}$ and $\mathbf{z}$ are disjoint.

  Select the perpendicular bisector of the line between $x_c$ and $z_c$ as the separating hyperplane.  %The norm of the hyperplane is ${w = x_c - z_c}$.  Select $b$ such that the midpoint between $x_c$ and $z_c$ is along the separator.  Therefore:

  \begin{aligncustom}
    w^{T}x + b &= 0 \\
    b &= -\left(x_c - z_c\right)^{T}\frac{x_c+z_c}{2}  \\
    b &= \frac{z_c \cdot z_c - x_c\cdot x_c}{2} = \frac{\norm{z_c}^2 - \norm{x_c}^{2}}{2} \text{.}
  \end{aligncustom}

\textit{By contradiction.}  Assume the above is not a separating hyperplane.  That means a point, $z_j\in\mathbf{z}$ is above this hyperplane.  Since $\mathbf{z}$ is convex, all points on the line,~$L$, between $z_c$ and $z_j$ must also be in $\mathbf{z}$.  $L$~is clearly not parallel to the hyperplane (since it crosses it).  This means there is necessarily a point on $L$ that is closer to~$x_c$ (the closest point is perpendicular to $L$ through $x_c$).  This is a contradiction because~$z_c$ and~$x_c$ are the closest pair of points.

The proof for $\mathbf{x}$ being above the separating hyperplane follows similarly.

\end{proof}

\noindent
\textbf{Continued on next page.}

\newpage
\begin{subproblem}
  (BONUS\@: Prove that two sets of points are linearly separable if and only if their convex hulls do not intersect.)
\end{subproblem}

In the first part of the problem, we proved that the convex hulls not intersecting implied the two sets are linearly separable.  To conclude ``if and only if,'' we must also prove that linear separability implies the convex hulls do not intersect, which is done below.

\begin{lemma}
  $\mathbf{z}$ and $\mathbf{x}$ linearly separable $\Rightarrow$ $\mathbf{x}$ \& $\mathbf{z}$ do not intersect
\end{lemma}

\begin{proof}
  \textit{By contradiction}.  Define $\mathbf{x}$ and $\mathbf{z}$ as above.

  Assume $\mathbf{x}$ and $\mathbf{z}$ are linearly separable but that $\mathbf{z}$ and $\mathbf{x}$ \textit{do} intersect, i.e.,~are not disjoint. Define $x_{z}$ such that $x_{z}\in\mathbf{x}$ and $x_z\in\mathbf{z}$.

  For the linearly separating hyperplane, $w^{T}x_{z}+b = c$ where $c$ is some constant.  Since $\mathbf{x}$ and $\mathbf{z}$ are linearly separable, $c>0$ and $c<0$ which is a contradiction.
\end{proof}

\end{document}

