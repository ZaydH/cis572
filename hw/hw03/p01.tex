\begin{problem}
An analyst wants to classify a number of customers based on some given attributes: total number of accounts, credit utilization (amount of used credit divided by total credit amount), percentage of on-time payments, age of credit history, and inquiries (number of time that the customer requested a new credit account, whether accepted or not). The analyst acquired some labeled information as shown in the following table:

\begin{table}[h]
  \centering
  \caption{Raw training data for problem~\#1}\label{tab:P01:RawTraining}
  \begin{tabular}{|c||c|c|c|c|c|c|}
    \hline
    ID & Total    & Utilization & Payment & Age of History & Inquiries & Label \\
       & Accounts &             & History & (days)         &           & \\\hline\hline
    1 & 8 & 15\% & 100\% & 1000 & 5 & GOOD \\\hline
    2 & 15 & 19\% & 90\% & 2500 & 8 & BAD \\\hline
    3 & 10 & 35\% & 100\% & 500 & 10 & BAD \\\hline
    4 & 11 & 40\% & 95\% & 2000 & 6 & BAD \\\hline
    5 & 12 & 10\% & 99\% & 3000 & 6 & GOOD \\\hline
    6 & 18 & 15\% & 100\% & 2000 & 5 & GOOD \\\hline
    7 & 3 & 21\% & 100\% & 1500 & 7 & BAD \\\hline
    8 & 14 & 4\% & 100\% & 3500 & 5 & GOOD \\\hline
    9 & 13 & 5\% & 100\% & 3000 & 3 & GOOD \\\hline
    10 & 6 & 25\% & 94\% & 2800 & 9 & BAD \\\hline
  \end{tabular}
\end{table}


Consider the following three accounts to be labeled:

\begin{table}[h]
  \centering
  \caption{Accounts to be labeled}\label{tab:P01:AccountsToLabel}
  \begin{tabular}{|c|c|c|c|c|c|}
    \hline
    Total    & Utilization & Payment & Age of History & Inquiries & Label \\
    Accounts &             & History & (days)         &           & \\\hline\hline
    20 & 50\% & 90\% & 4500 & 12 & P1 \\\hline
    8 & 10\% & 100\% & 550 & 4 & P2 \\\hline
    9 & 13\% & 99\% & 3000 & 6 & P3 \\\hline
  \end{tabular}
\end{table}
\end{problem}

\begin{subproblem}
  Before using nearest neighbor methods to make predictions, how would you recommend processing or transforming the data? Why? Make any changes you think appropriate to the data before continuing on to the next two parts.
\end{subproblem}

K-NN relies on a distance metric to quantify similarity between two examples.  If features are not changed to a consistent scales, necessarily some features will have higher weight than other.  In some cases, that may be advantageous but doing so blindly is dangerous and may yield dubious classifications.

The two most common approaches for transforming features are \textit{standardization} and \textit{normalization}.

\begin{table}[h]
  \centering
  \caption{Normalized training data for problem~\#1}\label{tab:P01:RawTraining}
  \begin{tabular}{|c||c|c|c|c|c|c|}
    \hline
    ID & Total    & Utilization & Payment & Age of History & Inquiries & Label \\
       & Accounts &             & History & (days)         &           & \\\hline\hline
    1  & 8   & 15\% & 100\% & 1000 & 5 & GOOD \\\hline
    2  & 15  & 19\% & 90\% & 2500 & 8 & BAD \\\hline
    3  & 10  & 35\% & 100\% & 500 & 10 & BAD \\\hline
    4  & 11  & 40\% & 95\% & 2000 & 6 & BAD \\\hline
    5  & 12  & 10\% & 99\% & 3000 & 6 & GOOD \\\hline
    6  & 18  & 15\% & 100\% & 2000 & 5 & GOOD \\\hline
    7  & 3   & 21\% & 100\% & 1500 & 7 & BAD \\\hline
    8  & 14  & 4\% & 100\% & 3500 & 5 & GOOD \\\hline
    9  & 13  & 5\% & 100\% & 3000 & 3 & GOOD \\\hline
    10 & 6   & 25\% & 94\% & 2800 & 9 & BAD \\\hline
  \end{tabular}
\end{table}

\begin{subproblem}
  What are the predicted labels P1, P2, and P3 using 1-NN with L1 distance? Assume that percentages are represented as their corresponding decimal numbers, so 95\% = 0.95. Show your work.
\end{subproblem}

\begin{subproblem}
  Keep the information of customers 7, 8, 9, and 10 as validation data, and find the best K value for the K-NN algorithm. If the best value of $K$ is not equal to 1, find the new predictions for P1, P2, and P3. Show your work.
\end{subproblem}


