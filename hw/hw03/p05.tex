\begin{problem}
Given a set of data points $x_1,\ldots,x_n$, we can define the convex hull to be the set of all points $x$ given by

\[x = \sum_{i=1}^{n} \alpha_i x_i \]

\noindent
where $\alpha_{i} \geq 0$ and $\sum_{i=1}^{n} \alpha_i = 1$. Consider a second set of points $z_1,\ldots,z_m$ together with their corresponding convex hull. By definition, the two sets of points will be linearly separable if there exists a vector $w$ and a scalar $b$ such that $w^{T}x_i + b > 0$ for all $x_i$ and $w^{T}z_i + b < 0$ for all $z_i$.

Prove that the two sets of points are linearly separable if their convex hulls do not intersect.
\end{problem}

\begin{lemma}
  $\mathbf{z}$ \& $\mathbf{x}$ do not intersect $\Rightarrow$ $\mathbf{z}$ and $\mathbf{x}$ linearly separable.
\end{lemma}

\begin{proof}

  Define $\mathbf{x}$ and $\mathbf{z}$ as the convex hulls enumerated above. Define $x_c$ and $z_c$ such that:

  \[ \argmin_{x_c \in \mathbf{x}, z_c \in \mathbf{z}} \norm{x_c - z_c} \text{.} \]

  \noindent
  This distance is strictly positive since $\mathbf{x}$ and $\mathbf{z}$ are disjoint.

  Select the perpendicular bisector of the line between $x_c$ and $z_c$ as the separating hyperplane.  %The norm of the hyperplane is ${w = x_c - z_c}$.  Select $b$ such that the midpoint between $x_c$ and $z_c$ is along the separator.  Therefore:

  \begin{aligncustom}
    w^{T}x + b &= 0 \\
    b &= -\left(x_c - z_c\right)^{T}\frac{x_c+z_c}{2}  \\
    b &= \frac{z_c \cdot z_c - x_c\cdot x_c}{2} = \frac{\norm{z_c}^2 - \norm{x_c}^{2}}{2} \text{.}
  \end{aligncustom}

\textit{By contradiction.}  Assume the above is not a separating hyperplane.  That means a point, $z_j\in\mathbf{z}$ is above this hyperplane.  Since $\mathbf{z}$ is convex, all points on the line,~$L$, between $z_c$ and $z_j$ must also be in $\mathbf{z}$.  $L$~is clearly not parallel to the hyperplane (since it crosses it).  This means there is necessarily a point on $L$ that is closer to~$x_c$ (the closest point is perpendicular to $L$ through $x_c$).  This is a contradiction because~$z_c$ and~$x_c$ are the closest pair of points.

The proof for $\mathbf{x}$ being above the separating hyperplane follows similarly.

\end{proof}

\noindent
\textbf{Continued on next page.}

\newpage
\begin{subproblem}
  (BONUS\@: Prove that two sets of points are linearly separable if and only if their convex hulls do not intersect.)
\end{subproblem}

In the first part of the problem, we proved that the convex hulls not intersecting implied the two sets are linearly separable.  To conclude ``if and only if,'' we must also prove that linear separability implies the convex hulls do not intersect, which is done below.

\begin{lemma}
  $\mathbf{z}$ and $\mathbf{x}$ linearly separable $\Rightarrow$ $\mathbf{x}$ \& $\mathbf{z}$ do not intersect
\end{lemma}

\begin{proof}
  \textit{By contradiction}.  Define $\mathbf{x}$ and $\mathbf{z}$ as above.

  Assume $\mathbf{x}$ and $\mathbf{z}$ are linearly separable but that $\mathbf{z}$ and $\mathbf{x}$ \textit{do} intersect, i.e.,~are not disjoint. Define $x_{z}$ such that $x_{z}\in\mathbf{x}$ and $x_z\in\mathbf{z}$.

  For the linearly separating hyperplane, $w^{T}x_{z}+b = c$ where $c$ is some constant.  Since $\mathbf{x}$ and $\mathbf{z}$ are linearly separable, $c>0$ and $c<0$ which is a contradiction.
\end{proof}
