\section{Experiments}

\subsection{}

Decision boundary margin is a common metric for measuring the quality of a binary classification algorithm. Example~$\xBase$ is labeled positive when ${\pHatDist<\nHatDist}$; otherwise $\xBase$~is labeled negative. Figure~\ref{fig:Experiments:UnlabelPlot} displays the positive and negative decoder losses for the MNIST experiment in Table~\ref{}.

Each mark in the graph represents a single training example where the shape and color indicate the mark's actual label and whether the example was in $\Pos$ or~$\Unlabel$. $\pHatDist$ is the $x$-axis while $\nHatDist$ is the $y$-axis. The decision boundary where ${\pHatDist=\nHatDist}$ is represented by the gray dashed line.  As previous described, the predicted label of any point of above this positive while those examples below the line are predicted negative.

Ideally, all positive-valued examples (shown as \red{XXXX}) would be in the upper left corner of the figure ---~${\pHatDist\ll\nHatDist}$~--- while negative-value examples would be in the lower right corner ---~${\pHatDist\gg\nHatDist}$.

\begin{figure}[h]
  \centering
  \newcommand{\scatterAxisMin}{0}
\newcommand{\scatterAxisMax}{0.30}
  \centering
  \begin{tikzpicture}
      \pgfplotstableread[col sep=comma] {plots/data/scatter_sep_data.csv}\thedata
      \begin{axis}
          [%xmode=log,
           %ymode=log,
           ymin=\scatterAxisMin,
           ymax=\scatterAxisMax,
           xmin=\scatterAxisMin,
           xmax=\scatterAxisMax,
           xlabel={\large $\pHatDist$},
           ylabel={\large $\nHatDist$},
           % legend style={at={(0.008,0.8)},anchor=west},
           point meta=explicit,
           xtick distance=0.05,
           ytick distance=0.05,
           % xtick={0.1,1,10,60,3600,36000},
           % xticklabels={$0.1\textrm{sec}$,$1\textrm{sec}$,$10\textrm{sec}$,$1\textrm{min}$,$1\textrm{hr}$,$10\textrm{hr}$},
           % ytick={0.1,1,10,60,3600,25200},
           % yticklabels={$0.1\textrm{sec}$,$1\textrm{sec}$,$10\textrm{sec}$,$1\textrm{min}$,$1\textrm{hr}$,$7\textrm{hr}$},
%           every tick label/.append style={font=\scriptsize},  % Reduce axis font size
           width=12cm,
           height=12cm,
%           axis x line*=bottom,  % Remove axis line from top
%           axis y line*=left,    % Remove axis line from right
           % axis line style={draw=none},
           % xtick pos=left, % Remove top ticks
           % ytick pos=left, % Remove right ticks
%           legend pos=north west,
%           legend pos=south east,
%           legend pos=outer north east,
           legend cell align=left,              % Align text left in legend
           legend image post style={scale=0.9},   % Increase marker size in legend
          ]
          \addplot[
                     scatter,
                     only marks,
                     scatter/classes={
                        1={mark=square*,color=black,very thin,mark size=1.65pt,fill opacity=0.3,{fill=red}},
                        2={mark=triangle*,fill opacity=0.3,color=black,mark size=2.4pt,{fill=green}},
                        0={mark=o,blue}
                     },
                     mark size=2pt
                   ]
                   table[x index=1,y index=2, meta index=0] {\thedata};
          \addplot[dashed, domain=\scatterAxisMin:\scatterAxisMax, smooth, color=gray] {x} {};
%          \addplot[dashed, domain=\scatterAxisMin:36000, smooth, color=gray] {x} node[very near start] (endofplotsquare) {};
%          \node [above left,color=black,thick] at (endofplotsquare) {$y=x$};
          % \draw [thick, color=black] (\scatterAxisMin,\scatterAxisMin) -- (\scatterAxisMax,\scatterAxisMax);  % vertical
          % \draw [color=black] (36000,\scatterAxisMin) -- (36000,25200);
          \legend{$\Unlabel\in\{\posLabel\}$,
                  $\Unlabel\in\{\negLabel\}$,
                  $\Pos$
                 }
      \end{axis}
  \end{tikzpicture}

  \caption{Separation of unlabeled examples}\label{fig:Experiments:UnlabelPlot}
\end{figure}

