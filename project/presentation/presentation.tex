\documentclass[11pt,dvipsnames,usenames,aspectratio=169]{beamer}  % Add handout to options to disable overlays

% For more themes, color themes and font themes, see:
% http://deic.uab.es/~iblanes/beamer_gallery/index_by_theme.html
%
\mode<presentation>
{%
  \usetheme{CambridgeUS}    % or try default, Darmstadt, Warsaw, ...
  \usecolortheme{whale}     % or try albatross, beaver, crane, ...
  \usefonttheme{serif}          % or try default, structurebold, ...
  % \usefonttheme[onlymath]{serif}
  % \setbeamertemplate{navigation symbols}{}
  % \setbeamercovered{transparent}

  \setbeamercolor{title}{fg=white}
  \setbeamerfont{title}{series=\bfseries}
  \setbeamercolor{frametitle}{fg=black}
  \setbeamerfont{frametitle}{series=\bfseries}

  \setbeamercolor{section in head/foot}{fg=white}
  \setbeamerfont{section in head/foot}{series=\bfseries}
  \setbeamercolor{subsection in head/foot}{fg=white}
  \setbeamerfont{subsection in head/foot}{series=\bfseries}
  \setbeamercolor{author in head/foot}{fg=white}
  \setbeamerfont{author in head/foot}{series=\bfseries}
  \setbeamercolor{title in head/foot}{fg=white}
  \setbeamerfont{title in head/foot}{series=\bfseries}

  \setbeamercolor{block title}{use=structure,fg=white,bg=title in head/foot.bg}
  \setbeamerfont{block title}{series=\bfseries}
  \setbeamercolor{block body}{use=structure,fg=black,bg=black!1!white}
}

% Support graying out frame elements
\newcommand{\FrameOpague}{\setbeamercovered{again covered={\opaqueness<1->{40}}}}
% Transition slide
\newcommand{\transitionFrame}[1]{%
{%
  \begin{frame}[plain,noframenumbering]{}{} % the plain option removes the sidebar and header from the title page
    \setbeamertemplate{final page}[text]{\Large \textbf{#1}}
    \usebeamertemplate{final page}
  \end{frame}}
}

% \usepackage{hyperref}     % Loaded automatically by beamer
\usepackage{pgfplots}       % Used to generate embedded plots
\pgfplotsset{compat=1.13}

% Imported via UltiSnips
\usepackage{mathtools} % for "\DeclarePairedDelimiter" macro
\DeclarePairedDelimiter{\floor}{\lfloor}{\rfloor}
\DeclarePairedDelimiter{\ceil}{\lceil}{\rceil}
\DeclarePairedDelimiter{\abs}{\lvert}{\rvert}
\DeclarePairedDelimiter{\norm}{\lVert}{\rVert}
% Imported via UltiSnips
\usepackage{amsmath}
\DeclareMathOperator*{\argmax}{arg\,max}
\DeclareMathOperator*{\argmin}{arg\,min}
\usepackage{amsfonts}  % Used for \mathbb and \mathcal
\usepackage{amssymb}
% Imported via UltiSnips
\usepackage{tikz}
\usetikzlibrary{arrows.meta,decorations.markings,shadows,positioning,calc,backgrounds,shapes,overlay-beamer-styles}

\newcommand{\pos}{\mathcal{P}}
\newcommand{\unlabel}{\mathcal{U}}

\usepackage{color}
\newcommand{\blue}[1]{{\color{Blue} #1}}
\newcommand{\red}[1]{{\color{red} #1}}
\newcommand{\green}[1]{{\color{ForestGreen} #1}}

%   Define all symbols that are used throughout the document
\newcommand{\toolname}{DeepPU}
\newcommand{\attLossBase}{ttractive loss}
\newcommand{\attLossCap}{A\attLossBase}
\newcommand{\attLossLow}{a\attLossBase}

% Imported via UltiSnips
\usepackage{color}
\newcommand{\colortext}[2]{{\color{#1} #2}}
\newcommand{\red}[1]{\colortext{red}{#1}}
\newcommand{\blue}[1]{\colortext{red}{#1}}
\newcommand{\green}[1]{\colortext{green}{#1}}

\newcommand{\etal}{~et~al.}
\newcommand{\elkan}{Elkan \&~Noto}
% Datasets
\newcommand{\CIFARten}{CIFAR\=/10}
\newcommand{\kmnist}{K\=/MNIST}   % Use unbreakable hyphen
\newcommand{\fashmnist}{Fashion\=/MNIST}   % Use unbreakable hyphen

\newcommand{\Pos}{\mathcal{P}}
\newcommand{\Neg}{\mathcal{N}}
\newcommand{\Unlabel}{\mathcal{U}}

% Attractive loss function names
\newcommand{\lAtt}{\mathcal{L}_{\text{Att}}}
\newcommand{\lAttBase}[1]{\mathcal{L}_{\text{Att-}#1}}
\newcommand{\lPosAtt}{\lAttBase{\Pos}}
\newcommand{\lUAtt}{\lAttBase{\Unlabel}}
\newcommand{\lPuBase}[1]{\mathcal{L}_{\text{PU-}#1}}
\newcommand{\lPu}{\lPuBase{\Pos{/}\Unlabel}}
\newcommand{\lPuP}{\lPuBase{\Pos}}
\newcommand{\lPuU}{\lPuBase{\Unlabel}}

% Labels for the positive and negative classes
\newcommand{\posLabel}{+1}
\newcommand{\negLabel}{-1}

\newcommand{\xBase}{\mathbf{x}}
\newcommand{\xDomain}{\mathcal{X}}
\newcommand{\lTrip}{\mathcal{L}_{\text{Triplet}}}
\newcommand{\SiamFunc}{f}
\newcommand{\exA}{\xBase_{a}}
\newcommand{\exP}{\xBase_{p}}
\newcommand{\exN}{\xBase_{n}}

% Symbols used in drawing of Deep PU
\newcommand{\xHatP}{\hat{\xBase}_{p}}
\newcommand{\xHatN}{\hat{\xBase}_{n}}
\newcommand{\zBase}{\mathbf{z}}
\newcommand{\zS}{\zBase_{s}}
\newcommand{\zP}{\zBase_{p}}
\newcommand{\zN}{\zBase_{n}}
% Function names for the DeepPU architecture
\newcommand{\fPU}{g}
\newcommand{\fPUenc}{\fPU_{\text{enc}}}
\newcommand{\fPUp}{\fPU_{p}}
\newcommand{\fPUn}{\fPU_{n}}

% Symbols related to prediction of DeepPU
\newcommand{\distSym}{\delta}
\newcommand{\siamDist}[3]{\puDist{#1\left(#2\right)}{#1\left(#3\right)}}
\newcommand{\puDist}[2]{\distSym\left(#1, #2\right)}
\newcommand{\pHatDist}{\puDist{\xBase}{\xHatP}}
\newcommand{\nHatDist}{\puDist{\xBase}{\xHatN}}
\newcommand{\puDistDiff}{\pHatDist - \nHatDist}
\newcommand{\sign}[1]{\text{sgn}\Big(#1\Big)}


% Here's where the presentation starts, with the info for the title slide
\title[Deep Positive-Unlabeled Learning]{Constructing a Positive-Unlabeled Learner using Deep Generative Models}
\author[Zayd Hammoudeh]{%
  \href{mailto:zayd@cs.uoregon.edu}{\textbf{Zayd Hammoudeh}}\inst{1}  % \textsuperscript{(\Letter)}
  % \and
  % \href{mailto:lowd@cs.uoregon.edu}{Daniel Lowd}\inst{1}
}

\institute[Univ.\ Oregon]{%
  \textsuperscript{1}\textbf{University of Oregon}\\
  Eugene, OR, USA\\
  \texttt{\href{mailto:zayd@cs.uoregon.edu}{zayd@cs.uoregon.edu}}
  % \texttt{{zayd, lowd}@ucsc.edu}
}
\date{\today}


\begin{document}

\begin{frame}
  \titlepage
\end{frame}

\begin{frame}{What is PU Learning?}
  \begin{itemize}[<+->]
    \setlength{\itemsep}{14pt}
    \item PU = Positive-Unlabeled
    \item Form of \blue{\textbf{binary classification}}
    \item Example of \textit{partially-supervised learning}
    \item \green{\textbf{Non-traditional}} training dataset $\mathcal{S} \coloneqq \pos \cup \unlabel$
      \begin{itemize}[<+->]
        \item $\pos$: Labeled examples all from positive class
        \item $\unlabel$: Unlabeled training set with \blue{\textbf{unknown distribution}} of positive \& negative examples
      \end{itemize}
    \item \textit{Example Applications}: Protein-similarity prediction, land-cover classification, targeted marketing, deceptive/incentivized review identification
      \begin{itemize}[<+->]
        \item Both \blue{inductive} \& \red{transductive} domains
        \item We will focus on the \red{transductive} learning
      \end{itemize}
  \end{itemize}
\end{frame}

\begin{frame}{Siamese Networks}
  \begin{itemize}
    \item
  \end{itemize}
\end{frame}

\begin{frame}{Our Architecture}
  \begin{center}
    \scalebox{0.65}{
\newcommand{\tikzVec}[4]{\node[vector, rotate=#4, node distance=2cm, #1] (#2) {\rotatebox{-#4}{#3}}}
\newcommand{\vertVec}[3]{\tikzVec{#1}{#2}{#3}{00}}
\newcommand{\zHeight}{1.1cm}
\newcommand{\zExtraHeight}{\zHeight/2}
\newcommand{\zNodeDist}{(\zHeight+\zExtraHeight)/2}
\newcommand{\zCurlXShift}{0.45cm}
\newcommand{\zAmplitude}{0.4cm}
\newcommand{\Pcolor}{black!60!blue}
\newcommand{\Ncolor}{black!60!red}

\begin{tikzpicture}[
    vector/.style = {%
      draw,
      rectangle,
      drop shadow,
      fill=white,
      minimum width=0.58cm,
      minimum height=2.1cm,
      font={\scriptsize},
    },
    ae trapezoid/.style = {%
      draw,
      trapezium,
      drop shadow,
      fill=white,
      trapezium angle=75,
      minimum width=1.8cm,
      minimum height=1cm,
      thick,
    },
    net line/.style = {%
      thick,
      -latex,
    },
    enc fill/.style={
      fill=orange!10,
    },
    enc fill on/.style={alt=#1{enc fill}{}},
    latent fill/.style={
      fill=yellow!10,
    },
    latent fill on/.style={alt=#1{latent fill}{}},
    pos dec fill/.style={
      fill=blue!10,
    },
    pos dec fill on/.style={alt=#1{pos dec fill}{}},
    neg dec fill/.style={
      fill=red!10,
    },
    neg dec fill on/.style={alt=#1{neg dec fill}{}},
  ]

  \vertVec{}{xi}{$\xBase$};
  \node[ae trapezoid, right of=xi, node distance=2.4cm, rotate=-90, enc fill on=<2-2>] (enc) {\rotatebox{90}{$\fPUenc(\cdot;\theta_{e})$}};
  \draw[net line] (xi) -- (enc);

  \coordinate[right of=enc, node distance=2.5cm] (zMid);
  \vertVec{above of=zMid, minimum height=\zExtraHeight, node distance=\zNodeDist, latent fill on=<3-3>}{zP}{$\zP$};
  \vertVec{above of=zMid, node distance=0cm, minimum height=\zHeight, latent fill on=<3-3>}{zS}{$\zS$};
  \vertVec{below of=zMid, minimum height=\zExtraHeight, node distance=\zNodeDist, latent fill on=<3-3>}{zN}{$\zN$};
  \draw[net line] (enc) -- (zS);

  \coordinate[above right=0cm and \zCurlXShift of zP.north] (zpCurlTop);
  \coordinate[above right=0cm and \zCurlXShift of zS.south] (zpCurlBot);
  \draw[decoration={brace,mirror,amplitude=\zAmplitude},decorate,thick,color=\Pcolor] (zpCurlBot) -- node[right=0.35*\zAmplitude] (zPout) {} (zpCurlTop);

  \coordinate[above right=0cm and \zCurlXShift of zN.south] (znCurlBot);
  \coordinate[above right=0cm and \zCurlXShift of zS.north] (znCurlTop);
  \draw[decoration={brace,mirror,amplitude=\zAmplitude},decorate,thick,color=\Ncolor] (znCurlBot) -- node[right=0.35*\zAmplitude] (zNout) {} (znCurlTop);

  \coordinate[right of=zMid, node distance=3.2cm] (decMid);
  \node[ae trapezoid, above of=decMid, node distance=1.5cm, rotate=90, pos dec fill on=<4-4>] (decP) {\rotatebox{-90}{$\fPUp(\cdot;\theta_{p})$}};
  \vertVec{right of=decP}{xHatP}{$\xHatP$};
  \draw[net line] (decP) -- (xHatP);

  \node[ae trapezoid, below of=decMid, node distance=1.5cm, rotate=90, neg dec fill on=<5-5>] (decN) {\rotatebox{-90}{$\fPUn(\cdot;\theta_{n})$}};
  \vertVec{right of=decN}{xHatN}{$\xHatN$};
  \draw[net line] (decN) -- (xHatN);

  \newcommand{\zOutNetLength}{0.7cm}
  \coordinate[right of=zPout, node distance=\zOutNetLength] (zpOutShift);
  \draw[net line,color=\Pcolor] (zPout) -- (zpOutShift) |- (decP);

  \coordinate[right of=zNout, node distance=\zOutNetLength] (znOutShift);
  \draw[net line,color=\Ncolor] (zNout) -- (znOutShift) |- (decN);
\end{tikzpicture}
}
  \end{center}

  \begin{itemize}
    \setlength{\itemsep}{10pt}
    \onslide<2->{\item Shared input encoder}
    \onslide<3->{\item Latent vector $\mathbf{z}$ partitioned into three parts}
    \onslide<4->{\item $g_{p}$: Decoder for \blue{positive}-examples}
    \onslide<5->{\item $g_{n}$: Decoder for \red{negative}-examples}
  \end{itemize}
\end{frame}

\begin{frame}{Experiments}
  \begin{itemize}
    \setlength{\itemsep}{16pt}
    \item \blue{\textbf{Dataset}}: MNIST
    \item \blue{\textbf{Baseline}}: Elkan \& Noto with Logistic Regression
    \item \blue{\textbf{Labeling Frequency}}: 50\%
  \end{itemize}
\end{frame}

\end{document}
