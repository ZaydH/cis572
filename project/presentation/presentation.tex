\documentclass[11pt,dvipsnames,usenames,aspectratio=169]{beamer}  % Add handout to options to disable overlays

% For more themes, color themes and font themes, see:
% http://deic.uab.es/~iblanes/beamer_gallery/index_by_theme.html
%
\mode<presentation>
{%
  \usetheme{CambridgeUS}    % or try default, Darmstadt, Warsaw, ...
  \usecolortheme{whale}     % or try albatross, beaver, crane, ...
  \usefonttheme{serif}          % or try default, structurebold, ...
  % \usefonttheme[onlymath]{serif}
  % \setbeamertemplate{navigation symbols}{}
  % \setbeamercovered{transparent}

  \setbeamercolor{title}{fg=white}
  \setbeamerfont{title}{series=\bfseries}
  \setbeamercolor{frametitle}{fg=black}
  \setbeamerfont{frametitle}{series=\bfseries}

  \setbeamercolor{section in head/foot}{fg=white}
  \setbeamerfont{section in head/foot}{series=\bfseries}
  \setbeamercolor{subsection in head/foot}{fg=white}
  \setbeamerfont{subsection in head/foot}{series=\bfseries}
  \setbeamercolor{author in head/foot}{fg=white}
  \setbeamerfont{author in head/foot}{series=\bfseries}
  \setbeamercolor{title in head/foot}{fg=white}
  \setbeamerfont{title in head/foot}{series=\bfseries}

  \setbeamercolor{block title}{use=structure,fg=white,bg=title in head/foot.bg}
  \setbeamerfont{block title}{series=\bfseries}
  \setbeamercolor{block body}{use=structure,fg=black,bg=black!1!white}
}

% Support graying out frame elements
\newcommand{\FrameOpague}{\setbeamercovered{again covered={\opaqueness<1->{40}}}}
% Transition slide
\newcommand{\transitionFrame}[1]{%
{%
  \begin{frame}[plain,noframenumbering]{}{} % the plain option removes the sidebar and header from the title page
    \setbeamertemplate{final page}[text]{\Large \textbf{#1}}
    \usebeamertemplate{final page}
  \end{frame}}
}

% \usepackage{hyperref}     % Loaded automatically by beamer
\usepackage{pgfplots}       % Used to generate embedded plots
\pgfplotsset{compat=1.13}

% Imported via UltiSnips
\usepackage{mathtools} % for "\DeclarePairedDelimiter" macro
\DeclarePairedDelimiter{\floor}{\lfloor}{\rfloor}
\DeclarePairedDelimiter{\ceil}{\lceil}{\rceil}
\DeclarePairedDelimiter{\abs}{\lvert}{\rvert}
\DeclarePairedDelimiter{\norm}{\lVert}{\rVert}
% Imported via UltiSnips
\usepackage{amsmath}
\DeclareMathOperator*{\argmax}{arg\,max}
\DeclareMathOperator*{\argmin}{arg\,min}
\usepackage{amsfonts}  % Used for \mathbb and \mathcal
\usepackage{amssymb}
% Imported via UltiSnips
\usepackage{tikz}
\usetikzlibrary{arrows.meta,decorations.markings,shadows,positioning,calc,backgrounds,shapes,overlay-beamer-styles}
% Packages used for the Convolutional NN drawing
\usepackage{graphicx}
\graphicspath{{./img/}}
\usepackage{color}
\usepackage{pgfplots}
\usepackage{pgf-umlsd}
\usepackage{ifthen}
% Imported via UltiSnips
\usepackage[noend]{algpseudocode}
\usepackage[Algorithm,ruled]{algorithm}
\algnewcommand\algorithmicforeach{\textbf{for each}}
\algdef{S}[FOR]{ForEach}[1]{\algorithmicforeach\ #1\ \algorithmicdo}

\newcommand{\pos}{\mathcal{P}}
\newcommand{\unlabel}{\mathcal{U}}

\usepackage{color}
\newcommand{\blue}[1]{{\color{Blue} #1}}
\newcommand{\red}[1]{{\color{red} #1}}
\newcommand{\green}[1]{{\color{ForestGreen} #1}}

%   Define all symbols that are used throughout the document
\newcommand{\toolname}{DeepPU}
\newcommand{\attLossBase}{ttractive loss}
\newcommand{\attLossCap}{A\attLossBase}
\newcommand{\attLossLow}{a\attLossBase}

% Imported via UltiSnips
\usepackage{color}
\newcommand{\colortext}[2]{{\color{#1} #2}}
\newcommand{\red}[1]{\colortext{red}{#1}}
\newcommand{\blue}[1]{\colortext{red}{#1}}
\newcommand{\green}[1]{\colortext{green}{#1}}

\newcommand{\etal}{~et~al.}
\newcommand{\elkan}{Elkan \&~Noto}
% Datasets
\newcommand{\CIFARten}{CIFAR\=/10}
\newcommand{\kmnist}{K\=/MNIST}   % Use unbreakable hyphen
\newcommand{\fashmnist}{Fashion\=/MNIST}   % Use unbreakable hyphen

\newcommand{\Pos}{\mathcal{P}}
\newcommand{\Neg}{\mathcal{N}}
\newcommand{\Unlabel}{\mathcal{U}}

% Attractive loss function names
\newcommand{\lAtt}{\mathcal{L}_{\text{Att}}}
\newcommand{\lAttBase}[1]{\mathcal{L}_{\text{Att-}#1}}
\newcommand{\lPosAtt}{\lAttBase{\Pos}}
\newcommand{\lUAtt}{\lAttBase{\Unlabel}}
\newcommand{\lPuBase}[1]{\mathcal{L}_{\text{PU-}#1}}
\newcommand{\lPu}{\lPuBase{\Pos{/}\Unlabel}}
\newcommand{\lPuP}{\lPuBase{\Pos}}
\newcommand{\lPuU}{\lPuBase{\Unlabel}}

% Labels for the positive and negative classes
\newcommand{\posLabel}{+1}
\newcommand{\negLabel}{-1}

\newcommand{\xBase}{\mathbf{x}}
\newcommand{\xDomain}{\mathcal{X}}
\newcommand{\lTrip}{\mathcal{L}_{\text{Triplet}}}
\newcommand{\SiamFunc}{f}
\newcommand{\exA}{\xBase_{a}}
\newcommand{\exP}{\xBase_{p}}
\newcommand{\exN}{\xBase_{n}}

% Symbols used in drawing of Deep PU
\newcommand{\xHatP}{\hat{\xBase}_{p}}
\newcommand{\xHatN}{\hat{\xBase}_{n}}
\newcommand{\zBase}{\mathbf{z}}
\newcommand{\zS}{\zBase_{s}}
\newcommand{\zP}{\zBase_{p}}
\newcommand{\zN}{\zBase_{n}}
% Function names for the DeepPU architecture
\newcommand{\fPU}{g}
\newcommand{\fPUenc}{\fPU_{\text{enc}}}
\newcommand{\fPUp}{\fPU_{p}}
\newcommand{\fPUn}{\fPU_{n}}

% Symbols related to prediction of DeepPU
\newcommand{\distSym}{\delta}
\newcommand{\siamDist}[3]{\puDist{#1\left(#2\right)}{#1\left(#3\right)}}
\newcommand{\puDist}[2]{\distSym\left(#1, #2\right)}
\newcommand{\pHatDist}{\puDist{\xBase}{\xHatP}}
\newcommand{\nHatDist}{\puDist{\xBase}{\xHatN}}
\newcommand{\puDistDiff}{\pHatDist - \nHatDist}
\newcommand{\sign}[1]{\text{sgn}\Big(#1\Big)}


% Here's where the presentation starts, with the info for the title slide
\title[Deep Positive-Unlabeled Learning]{Positive-Unlabeled Learning using a Deep Hybrid Generative/Discriminative Model}
\author[Zayd Hammoudeh]{%
  \href{mailto:zayd@cs.uoregon.edu}{\textbf{Zayd Hammoudeh}}\inst{1\textsuperscript{*}}  % \textsuperscript{(\Letter)}
  % \and
  % \href{mailto:lowd@cs.uoregon.edu}{Daniel Lowd}\inst{1}
}

\institute[Univ.\ Oregon]{%
  \textsuperscript{1}\textbf{University of Oregon}\\
  Eugene, OR, USA\\
  \texttt{\href{mailto:zayd@cs.uoregon.edu}{zayd@cs.uoregon.edu}}
  % \texttt{{zayd, lowd}@ucsc.edu}
}
\date{\today}


\begin{document}

\begin{frame}
  \titlepage

  % \vspace{20pt}
  \begin{center}
    \textsuperscript{*} Under the supervision of \textbf{Daniel Lowd}
  \end{center}
\end{frame}

\begin{frame}{What is PU Learning?}
  \begin{itemize}[<+->]
    \setlength{\itemsep}{14pt}
    \item PU = Positive-Unlabeled
    \item Form of \blue{\textbf{binary classification}}
    % \item Example of \textit{partially-supervised learning}
    \item \green{\textbf{Non-traditional}} training dataset $\mathcal{S} \coloneqq \pos \cup \unlabel$ such that $\pos \cap \unlabel = \emptyset$
      \begin{itemize}[<+->]
        \item $\pos$: Labeled examples all from positive class
        \item $\unlabel$: Unlabeled training set with \blue{\textbf{unknown distribution}} of positive \& negative examples
      \end{itemize}
    \item \textit{Example Applications}: %Protein-similarity prediction, land-cover classification, targeted marketing, deceptive/incentivized review identification
      \begin{itemize}[<+->]
        \item Both \blue{inductive} \& \red{transductive} domains
        \item We will focus on the \red{transductive} learning
      \end{itemize}
  \end{itemize}
\end{frame}

\begin{frame}{What is an Autoencoder?}
  \onslide<+->{Neural \textbf{Generative Model}: Reconstructs input (image) from compressed representation}

  % \onslide<+->{\vspace{8pt}Used for \textit{Representation Learning} --- typically dimensionality reduction}
  \vspace{14pt}
  % Convolutional Autoencoder
% Modified Version Of: https://github.com/jettan/tikz_cnn

\newcommand{\CaeImgHeight}{25mm}%
% Source: https://tex.stackexchange.com/questions/295100/partial-or-entire-image-blurring-in-tikz
%Parameters:
%- at position
%- blur parameter
%- image name
\newcommand\blurredimage[3]{%
  \node[opacity=1.0] (output image) at (#1) {\includegraphics[height=\CaeImgHeight]{#3}};
  \node[opacity=0.2] at (#1+ #2, #2) {\includegraphics[height=\CaeImgHeight]{#3}};
  \node[opacity=0.2] at (#1+-#2, #2) {\includegraphics[height=\CaeImgHeight]{#3}};
  \node[opacity=0.2] at (#1+-#2,-#2) {\includegraphics[height=\CaeImgHeight]{#3}};
  \node[opacity=0.2] at (#1+ #2,-#2) {\includegraphics[height=\CaeImgHeight]{#3}};
}%


\centering
\hspace{9in}\resizebox{\textwidth}{!}{%
  \begin{tikzpicture}%
    \draw[use as bounding box, transparent] (-1.8,-1.8) rectangle (17.2, 3.2);
    % Define the macro.
    % 1st argument: Height and width of the layer rectangle slice.
    % 2nd argument: Depth of the layer slice
    % 3rd argument: X Offset --> use it to offset layers from previously drawn layers.
    % 4th argument: Options for filldraw.
    % 5th argument: Text to be placed below this layer.
    % 6th argument: Y Offset --> Use it when an output needs to be fed to multiple layers that are on the same X offset.
    \newcommand{\networkLayer}[6]{%
      \def\a{#1} % Used to distinguish input resolution for current layer.
      \def\b{0.02} %
      \def\c{#2} % Width of the cube to distinguish number of input channels for current layer.
      \def\t{#3} % X offset for current layer.
      \def\d{#4} % Y offset for current layer.
      % Draw the layer body.
      \draw[line width=0.3mm](\c+\t,0,\d) -- (\c+\t,\a,\d) -- (\t,\a,\d);              % back plane
      \draw[line width=0.3mm](\t,0,\a+\d) -- (\c+\t,0,\a+\d) node[midway,below] {#6} -- (\c+\t,\a,\a+\d) -- (\t,\a,\a+\d) -- (\t,0,\a+\d); % front plane
      \draw[line width=0.3mm](\c+\t,0,\d) -- (\c+\t,0,\a+\d);
      \draw[line width=0.3mm](\c+\t,\a,\d) -- (\c+\t,\a,\a+\d);
      \draw[line width=0.3mm](\t,\a,\d) -- (\t,\a,\a+\d);
      % Recolor visible surfaces
      \filldraw[#5] (\t+\b,\b,\a+\d) -- (\c+\t-\b,\b,\a+\d) -- (\c+\t-\b,\a-\b,\a+\d) -- (\t+\b,\a-\b,\a+\d) -- (\t+\b,\b,\a+\d); % front plane
      \filldraw[#5] (\t+\b,\a,\a-\b+\d) -- (\c+\t-\b,\a,\a-\b+\d) -- (\c+\t-\b,\a,\b+\d) -- (\t+\b,\a,\b+\d);
      % Colored slice.
      \ifthenelse {\equal{#5} {}} %
      {} % Do not draw colored slice if #4 is blank.
      {\filldraw[#5] (\c+\t,\b,\a-\b+\d) -- (\c+\t,\b,\b+\d) -- (\c+\t,\a-\b,\b+\d) -- (\c+\t,\a-\b,\a-\b+\d);} % Else, draw a colored slice.
    } %
    \begin{scope}[shift={(5,0)}]
      \onslide<+->{
        % INPUT
        \node[] (input image) at (-3.75,0.5) {\includegraphics[height=\CaeImgHeight]{tikz/img/muffins.jpg}};
        \node[above of=input image, node distance=1.8cm] {\LARGE $\mathbf{x}$};
        \networkLayer{3.0}{0.03}{-0.5}{0.0}{color=gray!80}{} %
        % ENCODER
        \networkLayer{3.0}{0.1}{0.0}{0.0}{color=white}{}    %
        \networkLayer{2.5}{0.2}{0.4}{0.0}{color=white}{}    %
        \networkLayer{2.0}{0.4}{0.95}{0.0}{color=white}{}   %
        \networkLayer{1.5}{0.4}{1.5}{0.0}{color=white}{}    %
        % BOTTLENECK
        \networkLayer{1.0}{0.4}{2.3}{0.0}{color=red!40}{}    %
        % DECODER
        \networkLayer{1.5}{0.4}{3.3}{0.0}{color=white}{}    %
        \networkLayer{2.0}{0.4}{4.3}{0.0}{color=white}{}    %
        \networkLayer{2.5}{0.2}{5.3}{0.0}{color=white}{}    %
        \networkLayer{3.0}{0.1}{6.2}{0.0}{color=white}{}    %
        % OUTPUT
        \networkLayer{3.0}{0.05}{7.2}{0.0}{color=blue!20}{}  %
        \blurredimage{8.9,0.5}{0.10}{tikz/img/muffins.jpg}  %
        \node[above of=output image, node distance=1.8cm] {\LARGE $\mathbf{\hat{x}}$};
      }
      \onslide<+->{%
        \draw[dashed, blue, thick] (-1.8,3.4) rectangle (2.8,-1.5);
        \node[] () at (-0.3, 3.8) {\blue{\textbf{Encoder}}};
      }
      \onslide<+->{%
        \draw[dashed, color=dartmouthgreen, thick] (1.8,3.4) rectangle (7.5,-1.5);
        \node[] () at (5.4, 3.8) {\textbf{\color{dartmouthgreen} Decoder}};
      }
      \onslide<+->{%
        \draw[color=lust, thick] (1.8,3.4) rectangle (2.8,-1.5);
        \node[] () at (2.3, 3.8) {\textbf{\color{lust} Bottleneck}};
      }
    \end{scope}
  \end{tikzpicture}  %
}

  % \\
  % \onslide<+->{\centering\Large Objective Function $J = \min \norm{\bfx - \mathbf{\hat{x}}}$}
\end{frame}

\begin{frame}{Basic Idea}
  \begin{itemize}[<+->]
    \setlength{\itemsep}{20pt}
    \item \textbf{Recall}: PU Learning is Binary Classification
      \begin{itemize}[<+->]
        \item \textit{Example}: \blue{\textbf{Positive}}=Cats \& \red{\textbf{Negative}}=Dogs
      \end{itemize}
    \item \textit{Project Intuition}: Simultaneously train two autoencoders (AE):
      \begin{itemize}[<+->]
        \setlength{\itemsep}{6pt}
        \item \textit{Positive AE}: Only reconstruct \blue{positive} examples \onslide<+->{(images of \blue{cats})}
        \item \textit{Negative AE}: Only reconstruct \red{negative} examples \onslide<+->{(images of \red{dogs})}
      \end{itemize}

    \item \textbf{Classification}: Assign label matching whichever autoencoder reconstructs input better
  \end{itemize}
\end{frame}

\begin{frame}{Our Architecture}
  \begin{center}
    \scalebox{0.65}{
\newcommand{\tikzVec}[4]{\node[vector, rotate=#4, node distance=2cm, #1] (#2) {\rotatebox{-#4}{#3}}}
\newcommand{\vertVec}[3]{\tikzVec{#1}{#2}{#3}{00}}
\newcommand{\zHeight}{1.1cm}
\newcommand{\zExtraHeight}{\zHeight/2}
\newcommand{\zNodeDist}{(\zHeight+\zExtraHeight)/2}
\newcommand{\zCurlXShift}{0.45cm}
\newcommand{\zAmplitude}{0.4cm}
\newcommand{\Pcolor}{black!60!blue}
\newcommand{\Ncolor}{black!60!red}

\begin{tikzpicture}[
    vector/.style = {%
      draw,
      rectangle,
      drop shadow,
      fill=white,
      minimum width=0.58cm,
      minimum height=2.1cm,
      font={\scriptsize},
    },
    ae trapezoid/.style = {%
      draw,
      trapezium,
      drop shadow,
      fill=white,
      trapezium angle=75,
      minimum width=1.8cm,
      minimum height=1cm,
      thick,
    },
    net line/.style = {%
      thick,
      -latex,
    },
    enc fill/.style={
      fill=orange!10,
    },
    enc fill on/.style={alt=#1{enc fill}{}},
    latent fill/.style={
      fill=yellow!10,
    },
    latent fill on/.style={alt=#1{latent fill}{}},
    pos dec fill/.style={
      fill=blue!10,
    },
    pos dec fill on/.style={alt=#1{pos dec fill}{}},
    neg dec fill/.style={
      fill=red!10,
    },
    neg dec fill on/.style={alt=#1{neg dec fill}{}},
  ]

  \vertVec{}{xi}{$\xBase$};
  \node[ae trapezoid, right of=xi, node distance=2.4cm, rotate=-90, enc fill on=<2-2>] (enc) {\rotatebox{90}{$\fPUenc(\cdot;\theta_{e})$}};
  \draw[net line] (xi) -- (enc);

  \coordinate[right of=enc, node distance=2.5cm] (zMid);
  \vertVec{above of=zMid, minimum height=\zExtraHeight, node distance=\zNodeDist, latent fill on=<3-3>}{zP}{$\zP$};
  \vertVec{above of=zMid, node distance=0cm, minimum height=\zHeight, latent fill on=<3-3>}{zS}{$\zS$};
  \vertVec{below of=zMid, minimum height=\zExtraHeight, node distance=\zNodeDist, latent fill on=<3-3>}{zN}{$\zN$};
  \draw[net line] (enc) -- (zS);

  \coordinate[above right=0cm and \zCurlXShift of zP.north] (zpCurlTop);
  \coordinate[above right=0cm and \zCurlXShift of zS.south] (zpCurlBot);
  \draw[decoration={brace,mirror,amplitude=\zAmplitude},decorate,thick,color=\Pcolor] (zpCurlBot) -- node[right=0.35*\zAmplitude] (zPout) {} (zpCurlTop);

  \coordinate[above right=0cm and \zCurlXShift of zN.south] (znCurlBot);
  \coordinate[above right=0cm and \zCurlXShift of zS.north] (znCurlTop);
  \draw[decoration={brace,mirror,amplitude=\zAmplitude},decorate,thick,color=\Ncolor] (znCurlBot) -- node[right=0.35*\zAmplitude] (zNout) {} (znCurlTop);

  \coordinate[right of=zMid, node distance=3.2cm] (decMid);
  \node[ae trapezoid, above of=decMid, node distance=1.5cm, rotate=90, pos dec fill on=<4-4>] (decP) {\rotatebox{-90}{$\fPUp(\cdot;\theta_{p})$}};
  \vertVec{right of=decP}{xHatP}{$\xHatP$};
  \draw[net line] (decP) -- (xHatP);

  \node[ae trapezoid, below of=decMid, node distance=1.5cm, rotate=90, neg dec fill on=<5-5>] (decN) {\rotatebox{-90}{$\fPUn(\cdot;\theta_{n})$}};
  \vertVec{right of=decN}{xHatN}{$\xHatN$};
  \draw[net line] (decN) -- (xHatN);

  \newcommand{\zOutNetLength}{0.7cm}
  \coordinate[right of=zPout, node distance=\zOutNetLength] (zpOutShift);
  \draw[net line,color=\Pcolor] (zPout) -- (zpOutShift) |- (decP);

  \coordinate[right of=zNout, node distance=\zOutNetLength] (znOutShift);
  \draw[net line,color=\Ncolor] (zNout) -- (znOutShift) |- (decN);
\end{tikzpicture}
}
  \end{center}

  \blue{\textbf{Components}}:
  \begin{itemize}[<+->]
    \setlength{\itemsep}{4pt}
    \item Shared input encoder
    \item Latent vector $\mathbf{z}$ %partitioned into three parts
    \item $g_{p}$: Decoder for \blue{positive}-examples
    \item $g_{n}$: Decoder for \red{negative}-examples
  \end{itemize}
\end{frame}

\begin{frame}{Custom Loss Functions}
  ``\blue{\textbf{Attractive}}'' Loss for~$\pos$
  \begin{equation}\label{eq:Loss:AttP}
    \lPosAtt = \max\Big\{ \puDistDiff + \alpha, 0 \Big\}
  \end{equation}

  \vfill
  ``\green{\textbf{Pick-a-Side}}'' Loss for~$\unlabel$
  \begin{equation}\label{eq:Loss:AttU}
    \lUAtt = \max\Big\{ - \big\lvert\puDistDiff\big\rvert + \alpha, 0 \Big\}
  \end{equation}

  \vfill
  Unfortunately, there is no time now to discuss these, but we can in Q\&A if interested\ldots
\end{frame}

\begin{frame}{Training Algorithm (Sketch)}
  \begin{algorithmic}[1]
  \Function{\toolname training}{$\pos,\unlabel$}
    \State Train $\fPUenc$ \& $\fPUn$ as an AE over $\unlabel$ \Comment{$\fPUn$ decodes positive \& negative classes}
    \State Freeze all weights in $\fPUenc$ except $\zP$
    \State Train $\fPUenc$ \& $\fPUp$ as an AE over $\pos$ \Comment{$\fPUp$ decodes positives only}
    \State Unfreeze all weights in $\fPUenc$
    \Statex
    \While{\text{not converged}}  \Comment{Force $\fPUn$ to forget how to reconstruct $\pos$}
      % \State Increment value of $\alpha$ \Comment{Increasing temperature parameter}
      \While{\text{epoch not complete}}
        \State Select batch $b_{\Pos}$ from $\Pos$
        \State Update $\vec{\theta}$ via $\nabla\lPuP(b_{\Pos})$
        \State Select batch $b_{\Unlabel}$ from $\Unlabel$
        \State Update $\vec{\theta}$ via $\nabla\lPuU(b_{\Unlabel})$
      \EndWhile
    \EndWhile
  \EndFunction
\end{algorithmic}

\end{frame}

\begin{frame}{Experiments}
  \begin{itemize}
    \setlength{\itemsep}{16pt}
    \item \blue{\textbf{Dataset}}: MNIST
      \begin{itemize}
        \item \textit{Positive Class}: $4$
        \item \textit{Negative Class}: $9$
      \end{itemize}
    \item \blue{\textbf{Baseline}}: Elkan \& Noto with Logistic Regression
    \item \blue{\textbf{Dataset Sizes}}:
      \begin{itemize}
        \item $\abs{\pos}$: ${\sim}$3,000
        \item $\abs{\unlabel}$: ${\sim}$6,000 evenly split between positive \& negative classes
      \end{itemize}
    \item \blue{\textbf{Labeling Frequency}}: 50\%
  \end{itemize}

  \vspace{16pt}
  Let's look how an example matches our earlier intuition\ldots
\end{frame}

\begin{frame}{Decoder Outputs --- Epoch \red{0}}
  \begin{columns}
    \begin{column}{0.59\textwidth}
      \begin{center}
        \includegraphics[scale=0.44]{deep-pu_epoch=000.jpg}
      \end{center}
    \end{column}
    \begin{column}{0.41\textwidth}
      \begin{itemize}[<+->]
        \setlength{\itemsep}{20pt}
        \item Positive decoder reconstructs everything as~4
        \item Negative decoder reconstructs both~4 and~9
      \end{itemize}
    \end{column}
  \end{columns}
\end{frame}

\begin{frame}{Decoder Outputs --- Epoch \red{1}}
  \begin{columns}
    \begin{column}{0.59\textwidth}
      \begin{center}
        \includegraphics[scale=0.44]{deep-pu_epoch=001.jpg}
      \end{center}
    \end{column}
    \begin{column}{0.41\textwidth}
      \begin{itemize}[<+->]
        \setlength{\itemsep}{20pt}
        \item Not much change after one epoch
      \end{itemize}
    \end{column}
  \end{columns}
\end{frame}

\begin{frame}{Decoder Outputs --- Epoch \red{10}}
  \begin{columns}
    \begin{column}{0.59\textwidth}
      \begin{center}
        \includegraphics[scale=0.44]{deep-pu_epoch=010.jpg}
      \end{center}
    \end{column}
    \begin{column}{0.41\textwidth}
      \begin{itemize}[<+->]
        \setlength{\itemsep}{20pt}
        \item Positive decoder still unchanged
        \item Negative decoder starting to close \blue{$4$}s to form \red{$9$}s
      \end{itemize}
    \end{column}
  \end{columns}
\end{frame}

\begin{frame}{Decoder Outputs --- Epoch \red{20}}
  \begin{columns}
    \begin{column}{0.59\textwidth}
      \begin{center}
        \includegraphics[scale=0.44]{deep-pu_epoch=020.jpg}
      \end{center}
    \end{column}
    \begin{column}{0.41\textwidth}
      \begin{itemize}[<+->]
        \setlength{\itemsep}{20pt}
        \item Positive decoder still unchanged
        \item Negative decoder turning \blue{$4$}s into \red{$9$}s even more
      \end{itemize}
    \end{column}
  \end{columns}
\end{frame}

\begin{frame}{Decoder Outputs --- Epoch \red{50}}
  \begin{columns}
    \begin{column}{0.59\textwidth}
      \begin{center}
        \includegraphics[scale=0.44]{deep-pu_epoch=050.jpg}
      \end{center}
    \end{column}
    \begin{column}{0.41\textwidth}
      \begin{itemize}[<+->]
        \setlength{\itemsep}{20pt}
        \item Positive decoder still unchanged
        \item Negative decoder almost entirely closes input \blue{$4$}s
      \end{itemize}
    \end{column}
  \end{columns}
\end{frame}

\begin{frame}{Decoder Outputs --- Epoch \red{100}}
  \begin{columns}
    \begin{column}{0.59\textwidth}
      \begin{center}
        \includegraphics[scale=0.44]{deep-pu_epoch=050.jpg}
      \end{center}
    \end{column}
    \begin{column}{0.41\textwidth}
      \begin{itemize}[<+->]
        \setlength{\itemsep}{20pt}
        \item Positive \& negative decoders always reconstruct \blue{$4$}s \& \red{$9$}s respectively
        \item Suitable to use the generative model as a discriminative classifier
      \end{itemize}
    \end{column}
  \end{columns}
\end{frame}

\begin{frame}{Quantitative Results}
  \begin{itemize}
    \setlength{\itemsep}{15pt}
    \onslide<+->{
      \item \blue{\textbf{Our Architecture}}:
        \begin{itemize}
          \item \textit{Accuracy}: 98.0\%
          \item \textit{AUC ROC}: 0.996
          \item \textit{F1 Score}: 0.980
        \end{itemize}
    }
    \onslide<+->{
      \item \green{\textbf{Elkan \& Noto}}: Using Logistic Regression
        \begin{itemize}
          \item Discussion of this algorithm is beyond the scope of this talk.
          \item \textit{Accuracy}: 89.9\%
          \item \textit{AUC ROC}: 0.969
          \item \textit{F1~Score}: 0.932
        \end{itemize}
    }
  \end{itemize}
\end{frame}

\begin{frame}[allowframebreaks]
  {\tiny
    \frametitle{References}
    \bibliographystyle{ieeetr}
    \bibliography{bib/ref.bib}
  }
\end{frame}

\appendix
\begin{frame}{Previous Work}
  \onslide<+->{\textbf{Two Primary Paradigms}}
  \begin{itemize}[<+->]
    \setlength{\itemsep}{16pt}
    \item ``\blue{\textbf{One Stage Approach}}''
      \begin{itemize}[<+->]
        \setlength{\itemsep}{4pt}
      \item \textit{Heuristically} extract ``reliable negative'' examples~$\mathcal{N}'\in\unlabel$
        \item Train binary classifier $\pos$ vs.\ $\mathcal{N}'$
        \item \red{Disadvantage}: Heuristic-based with high variance based on $\mathcal{N}'$ extraction
      \end{itemize}

    \item ``\blue{\textbf{Two Stage Approach}}''
      \begin{itemize}[<+->]
        \setlength{\itemsep}{4pt}
        \item Cost-sensitive optimization framework
        \item Assume all examples in~$\mathcal{U}$ are negative with a label weight proportional to confidence sample is negative
        \item Train binary classifier $\pos$ vs.\ $\unlabel$
      \end{itemize}
  \end{itemize}

  \vspace{10pt}
  \onslide<+->{\green{\textbf{Baseline for Comparison}}: Seminal two stage approach from Elkan \& Noto~\cite{Elkan:2008}}
\end{frame}

\end{document}
